\documentclass[a4paper,12pt]{article}

%\usepackage{times}
%\usepackage[T3,T1]{fontenc}
%\usepackage[utf8x]{inputenc}
% \usepackage[T1]{fontenc}
% \usepackage{mathptmx}  

\usepackage{fontspec} %,xunicode} % xltxtra
    \defaultfontfeatures{Ligatures=TeX}
    \setmainfont{Brill}[
    Extension=.ttf,
    UprightFont=*-Roman,
    BoldFont=*-Bold,
    ItalicFont=*-Italic,
    BoldItalicFont=*-Bold-Italic,
    Renderer=ICU]
% 	\usefonttheme{serif}
%	\setmainfont[Renderer=ICU]{Charis SIL}
% 	\setmainfont[Renderer=ICU]{Brill}

%% Trigger answers and notes. Package by Byron Ahn with edits by Craig Sailor
\usepackage[spaces]{myProbSol} %   Optional arguments:
%                                               'solution': reveals solutions
%                                               'spaces': leaves whitespace corresponding to the size of the solutions (has no effect if you also pass 'solution')


\usepackage{amsmath,amssymb} % for $\text{}$
\usepackage{url,natbib} 
\usepackage[dvipsnames]{xcolor} %,bm}
\usepackage{geometry,vmargin,setspace}
\usepackage{multirow}
\setmarginsrb{1in}{1in}{1in}{1in}{13.6pt}{0.1in}{0.1in}{0.2in}
% pdflscape} %rotating
    \setlength{\parindent}{0pt}

\usepackage{stmaryrd}
\usepackage{wasysym} %checkbox
\usepackage[normalem]{ulem} % for \sout{}
% \usepackage{mdwlist} % for \begin{itemize*} - but prefer \tightlist
\providecommand{\tightlist}{%
	\setlength{\itemsep}{0pt}\setlength{\parskip}{0pt}}

\usepackage{natbib}
	\bibpunct[:]{(}{)}{;}{a}{}{,}
	\setlength{\bibsep}{0pt plus 0.3ex}

\usepackage{longtable}
\usepackage[linkcolor=purple,citecolor=ForestGreen,colorlinks=true,urlcolor=gray,pagebackref=false]{hyperref}
\usepackage{multicol}
\usepackage{dashrule} %\hdashrule
\usepackage{array} % >{\...} in tabulars
\usepackage{arydshln}

\usepackage[framemethod=tikz,footnoteinside=false]{mdframed} 

\usepackage{expex} %Linguistics examples
\lingset{aboveexskip=0ex,belowexskip=0.5ex,aboveglftskip=-0.3em,interpartskip=0ex,labelwidth=!6pt,belowpreambleskip=0.1ex} %, *=*?}
%	\lingset{aboveexskip=0.5ex,belowexskip=0.5ex,*=??}

\usepackage[nocenter]{qtree} % trees
\usepackage{tikz}
    \tikzstyle{every picture}+=[remember picture]

\usepackage{tipa} % convenient for \textsubarch etc.

\newcommand\trace{\rule[-0.5ex]{0.5cm}{.4pt}}
\bibpunct[:]{(}{)}{;}{a}{}{,}
	\setlength{\bibsep}{0pt plus 0.3ex}

\usepackage{pifont}
\newcommand{\cmark}{\ding{51}}% 52
\newcommand{\xmark}{\ding{55}}%
 \newcommand{\hand}{\ding{43}}

\newcommand\zero{\O{}}
\newcommand\itp[1]{\textit{\textipa{#1}}}
\newcommand\gsc[1]{\textsc{\lowercase{#1}}} %for glossing in small caps - comment out to return to caps
\renewcommand\root[1]{$\sqrt{\text{#1}}$}

\newcommand\blue[1]{\textcolor{blue}{#1}}
\newcommand\red[1]{\textcolor{red}{#1}}
\newcommand\green[1]{\textcolor{ForestGreen}{#1}}
\newcommand\gray[1]{\textcolor{gray}{#1}}
\newcommand\denote[1]{$\llbracket$#1$\rrbracket$}
\newcommand\lra{$\leftrightarrow$}

\newcommand\vz{\text{Voice$_{\text{\{--D\}}}$}}
\newcommand\vd{\text{Voice$_{\text{\{+D\}}}$}}
\newcommand\pz{\text{$p_{\text{\zero}}$}}
\newcommand\va{\root{\gsc{ACTION}}}
\newcommand{\tkal}{\emph{XaYaZ}}
\newcommand{\tpie}{\emph{XiY̯eZ}}
\newcommand{\tpua}{\emph{XuY̯aZ}}
\newcommand{\thif}{\emph{heXYiZ}}
\newcommand{\thuf}{\emph{huXYaZ}}
\newcommand{\thit}{\emph{hitXaY̯eZ}}
\newcommand{\tnif}{\emph{niXYaZ}}
\newcommand\dgs[1]{\textsubarch{#1}}
\newcommand\del[1]{\sout{#1}}

\makeatletter %Allow superscript ^ and subscript _
\catcode`_=\active%
\gdef_#1{\ensuremath{{}\sb{#1}}}%
\catcode`^=\active%
\gdef^#1{\ensuremath{{}\sp{#1}}}%
\makeatother

\begin{document}
% \pagenumbering{gobble}
% \singlespacing


\hfill \emph{Morphology 2024-25, Edinburgh, Handout 8}
\bigskip

    \section{Causativization}
        \subsection{Thinking structurally}
[Jigsaw puzzle: how do we express causation? What are some of the kinds of \emph{causative verbs} that we find?]

% What do we think is going on in each case?
% \pex
%     \a Destroy the tower. (transitive/causative)
%     \a Open the door. (labile/alternating)
%     \a Make Sam dance. (periphrastic)
% \xe

We have seen examples of:
\pex 
    \a \inSolution{``Lexical'' causatives, where the combination of verb and causative suffix results in a novel, non-transparent verb.}
    \a \inSolution{``Analytical'' or ``syntactic'' causatives where the causee is also the object/patient.}
    \a \inSolution{``Analytical'' or ``syntactic'' causatives where the causee is also the subject/agent.}
\xe

How can these be represented formally?

\pex
    \inSolution{
    \a Lexical causatives, like all our idiosyncratic verbs, might be the result of adjoining \gsc{CAUS} to the root:\\
        \Tree
            [.
                [.\root{\gsc{smell}} ]
                [.{\gsc{CAUS}\\\emph{sase}} ]
            ]
    \a An analytical causative where the causee is the patient might only take the vP, so the argument remains the patient:\\
        \Tree
        [.
            [.{DP1\\Agent\\Causer} ]
            [.
                [.\gsc{CAUS} ]
                [.vP
                    [.v
                        [.\root{\gsc{root}} ]
                        [.v ]
                    ]
                    [.{DP2\\Theme\\Causee} ]
                ]
            ]
        ]
    \a An analytical causative where the causee is the agent might take a full VoiceP:\\
        \Tree
        [.
            [.{DP1\\Agent\\Causer} ]
            [.
                [.\gsc{CAUS} ]
                [.VoiceP
                    [.{DP2\\Agent\\Causee} ]
                    [.
                        [.Voice ]
                        [.vP 
                            [.v
                                [.\root{\gsc{ROOT}} ]
                                [.v ]
                            ]
                            [.{DP3\\Theme} ]
                        ]
                    ]
                ]
            ]
        ]
    }
\xe

We should now see what these analyses predict, and we have just the tool for that: sublexical modification.

        \subsection{Sublexical modification with causation}
Turkish, scope of \emph{again}:
\ex \begingl
    \gla Ö\v{g}retmen Mary-yi yine ko\c{s}-tur-du//
    \glb teacher Mary-\gsc{ACC} again run-\gsc{CAUS-PAST}//
    \glft `The teacher made Mary run again.'\\
    a. [\emph{again} \gsc{CAUS}] > V\\
    Context: \inSolution{It is sports day at school. Mary wanted to play volleyball in the morning, but the teacher made her run instead. In the afternoon the teacher made Mary run again.}\\
    b. \gsc{CAUS} > [\emph{again} V]\\
    Context: \inSolution{Mary ran around the field in the morning, but the teacher wasn't watching. So in the afternoon the teacher made Mary run again.}//
    \endgl
\xe

Turkish, scope of negation:
\ex \begingl
    \gla John Mary-yi koş-tur-ma-d\i//
    \glb John Mary-\gsc{ACC} run-\gsc{CAUS-NEG-PAST}//
    \glft a. \inSolution{`John did not make Mary run.' (= she ran on her own accord)}\\
    b. \inSolution{`John made Mary not run.' (= he prevented her from running)}//
    \endgl
\xe

Japanese, scope of disjunction `or':
\pex
    \a \begingl
        \gla Hanako-ga [Masao-ni [uti-o soozisuru]-ka [heya-dai-o haraw]]-aseru koto ni sita//
        \glb Hanako-\gsc{Nom} [Masao-\gsc{Dat} [house-\gsc{Acc} clean]-or [room-rent-\gsc{Acc} pay]]-\gsc{CAUS} that of do//
        \glft `Hanako decided to make Masao clean the house or pay room rent.'\\
        Reading: \gsc{CAUS} scopes over OR; Masao has a choice.//
    \endgl
    \a \begingl
        \gla Hanako-ga [Masao-ni [uti-o soozis-aseru]-ka [heya-dai-o haraw-aseru]] koto ni sita//
        \glb Hanako-\gsc{Nom} Masao-\gsc{Dat} house-\gsc{Acc} clean-\gsc{CAUS}-or room-rent-\gsc{Acc} pay-\gsc{CAUS} that of do//
        \glft `Hanako decided to make Masao clean the house or she decided to make him pay room rent.'\\
        Reading: OR scopes over \gsc{CAUS}; Masao won’t have a choice.//
    \endgl
\xe


Turkish, attachment of an agent-oriented adverb:
\pex \begingl
    \gla Anne-si Ay\c{s}e-ye sessizce Mary-yi oyna-t-tır-dı//
    \glb mother-3\gsc{SG.POSS} Ay\c{s}e-\gsc{DAT} quietly Mary-\gsc{ACC} play-\gsc{CAUS-CAUS-PAST}//
    \glft `Her mother made Ay\c{s}e make Mary play quietly.'//
    \endgl
    \a \emph{quietly} \gsc{CAUS} > \gsc{CAUS} > V\\
    Context: We are at a dinner party at Ay\c{s}e's house. Ay\c{s}e’s friend Mary is bored. Ay\c{s}e’s mother quietly asks Ay\c{s}e to make Mary play with her toys.
    \a \gsc{CAUS} > \emph{quietly} \gsc{CAUS} > V\\
    Context: Ay\c{s}e’s friend Mary is bored in the next room. Ay\c{s}e’s mother asks Ay\c{s}e to quietly go and make Mary play with her toys.
    \a \ljudge{\#} \gsc{CAUS} > \gsc{CAUS} > \emph{quietly} V\\
    Context: Ay\c{s}e’s friend Mary is playing loudly with her toys. Ay\c{s}e’s mother asks
    Ay\c{s}e to make Mary play quietly.
\xe

    \subsection{Summary}
Once we view morphological structure as the same thing as syntactic structure, we can make sense of how affixation influences (a) the interpretation of clauses (like with \emph{again}), and (b) which arguments get added.

We started off by observing that in English (and in other languages) a predicate might have an external argument, an internal argument and an applied argument - but that's it. Our formal system can derive this behavior if each argument is introduced (or sometimes people say \emph{licensed by}) one particular head: Voice for external arguments and v/V for internal arguments. For applied arguments, a very brief overview of the head Appl is given below (why can't we add multiple Appls? See \citealt{nie20phd} for one proposal).

\paragraph{Readings.} Start with \cite{woodmyler19} for argument structure, followed by \cite{marantz13lingua}, which makes some more specific technical claims. You might now want to delve deeper into \cite{kratzer96}, \cite{layering15}, or chapter 6 of \cite{kastner20ogs}.

For causatives in particular, see \cite{harley08,harley17oup}. There is a lot of exciting work on causatives nowadays; see \cite{nie20} and \cite{akkus19jl}, for example.

% \newpage
\bibliographystyle{linquiry2}
\bibliography{lingxbib}

\end{document}