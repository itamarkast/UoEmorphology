\documentclass[a4paper,12pt]{article}

%\usepackage{times}
%\usepackage[T3,T1]{fontenc}
%\usepackage[utf8x]{inputenc}
% \usepackage[T1]{fontenc}
% \usepackage{mathptmx}  

\usepackage{fontspec} %,xunicode} % xltxtra
    \defaultfontfeatures{Ligatures=TeX}
    \setmainfont{Brill}[
    Extension=.ttf,
    UprightFont=*-Roman,
    BoldFont=*-Bold,
    ItalicFont=*-Italic,
    BoldItalicFont=*-Bold-Italic,
    Renderer=ICU]
% 	\usefonttheme{serif}
%	\setmainfont[Renderer=ICU]{Charis SIL}
% 	\setmainfont[Renderer=ICU]{Brill}

\usepackage{amsmath,amssymb} % for $\text{}$
\usepackage{url,natbib} 
\usepackage[dvipsnames]{xcolor} %,bm}
\usepackage{geometry,vmargin,setspace}
\usepackage{multirow}
\setmarginsrb{1in}{1in}{1in}{1in}{13.6pt}{0.1in}{0.1in}{0.2in}
% pdflscape} %rotating
    \setlength{\parindent}{0pt}

\usepackage{stmaryrd}
\usepackage{wasysym} %checkbox
\usepackage[normalem]{ulem} % for \sout{}
% \usepackage{mdwlist} % for \begin{itemize*} - but prefer \tightlist
\providecommand{\tightlist}{%
	\setlength{\itemsep}{0pt}\setlength{\parskip}{0pt}}

\usepackage{natbib}
	\bibpunct[:]{(}{)}{;}{a}{}{,}
	\setlength{\bibsep}{0pt plus 0.3ex}

\usepackage{longtable}
\usepackage[linkcolor=purple,citecolor=ForestGreen,colorlinks=true,urlcolor=gray,pagebackref=true]{hyperref}
\usepackage{multicol}
\usepackage{dashrule} %\hdashrule
\usepackage{array} % >{\...} in tabulars
\usepackage{arydshln}

\usepackage[framemethod=tikz,footnoteinside=false]{mdframed} 

\usepackage{expex} %Linguistics examples
\lingset{aboveexskip=0ex,belowexskip=0.5ex,aboveglftskip=-0.3em,interpartskip=0ex,labelwidth=!6pt,belowpreambleskip=0.1ex} %, *=*?}
%	\lingset{aboveexskip=0.5ex,belowexskip=0.5ex,*=??}

\usepackage[nocenter]{qtree} % trees
\usepackage{tikz}
    \tikzstyle{every picture}+=[remember picture]

\usepackage{tipa} % convenient for \textsubarch etc.

\newcommand\trace{\rule[-0.5ex]{0.5cm}{.4pt}}
\bibpunct[:]{(}{)}{;}{a}{}{,}
	\setlength{\bibsep}{0pt plus 0.3ex}

\usepackage{pifont}
\newcommand{\cmark}{\ding{51}}% 52
\newcommand{\xmark}{\ding{55}}%
 \newcommand{\hand}{\ding{43}}

\newcommand\zero{\O{}}
\newcommand\itp[1]{\textit{\textipa{#1}}}
\newcommand\gsc[1]{\textsc{\lowercase{#1}}} %for glossing in small caps - comment out to return to caps
\renewcommand\root[1]{$\sqrt{\text{#1}}$}

\newcommand\blue[1]{\textcolor{blue}{#1}}
\newcommand\red[1]{\textcolor{red}{#1}}
\newcommand\green[1]{\textcolor{ForestGreen}{#1}}
\newcommand\gray[1]{\textcolor{gray}{#1}}
\newcommand\denote[1]{$\llbracket$#1$\rrbracket$}
\newcommand\lra{$\leftrightarrow$}

\newcommand\vz{\text{Voice$_{\text{\{--D\}}}$}}
\newcommand\vd{\text{Voice$_{\text{\{+D\}}}$}}
\newcommand\pz{\text{$p_{\text{\zero}}$}}
\newcommand\va{\root{\gsc{ACTION}}}
\newcommand{\tkal}{\emph{XaYaZ}}
\newcommand{\tpie}{\emph{XiY̯eZ}}
\newcommand{\tpua}{\emph{XuY̯aZ}}
\newcommand{\thif}{\emph{heXYiZ}}
\newcommand{\thuf}{\emph{huXYaZ}}
\newcommand{\thit}{\emph{hitXaY̯eZ}}
\newcommand{\tnif}{\emph{niXYaZ}}
\newcommand\dgs[1]{\textsubarch{#1}}
\newcommand\del[1]{\sout{#1}}

\makeatletter %Allow superscript ^ and subscript _
\catcode`_=\active%
\gdef_#1{\ensuremath{{}\sb{#1}}}%
\catcode`^=\active%
\gdef^#1{\ensuremath{{}\sp{#1}}}%
\makeatother

\begin{document}
% \pagenumbering{gobble}
% \singlespacing

\section*{What is a word, inflection vs derivation, etc}

See ``EdinMorph meeting records'' doc.


\section*{Warm-up}

Segment the following words.

\pex
    \a walk
    \a walks
    \a talk
    \a talks
    \a walking
    \a walkable
    \a walkability
\xe

And what issues arise when we segment these ones?

\pex
    \a bassoon
    \a balloon
    \a buffoon
    \a baboon
\xe
% Based on some of Jonathan Bobaljik's materials, not sure where it originated!

\pex
    \a glow
    \a glisten
    \a glimmer
\xe

Not every string that repeats itself is a grammatical primitive (a morpheme). How can we tell? What are we looking for? What kinds of evidence are relevant?

\section{Affix ordering}

    \subsection{Auxiliaries and preliminaries}
What's the order of auxiliaries in English?

\pex
    \a She will have be-en winn-ing the race.
    \a \ljudge{*} She has been will winning the race.
\xe
\pex
    \a The cake will have be-en (be-ing) eat-en.
    \a \ljudge{*} The cake is having were been eat.
\xe

T > Perf > Prog > Pass > V.

\bigskip
Each element:
\begin{itemize} \tightlist
    \item Takes semantic \emph{scope} over the next: the future of the perfect of the passive of an eating event.
    \item Takes morpho(-phono)logical \emph{scope} over the next: \emph{have}_{\gsc{perf}} triggers participial morphology on \emph{eat-en}, \emph{is}_{\gsc{prog}} triggers progressive morphology on \emph{winn-ing}, etc.
\end{itemize}

We've already talked about how to represent this formally:
\ex
\Tree
[.TP
    [.\emph{The cake} ]
    [.
        [.T\\\emph{will} ]
        [.PerfP/AspP
            [.Perf\\\emph{have} ]
            [.ProgP
                [.Prog\\\emph{been} ]
                [.PassP
                    [.Pass\\\emph{being} ]
                    [.VP\\\emph{eaten} ]
                ]
            ]
        ]
    ]
]
\xe

What's the order in Latin \citep{embick10,kastnerzu17,kastner18nllt}?
\pex
    \a \begingl
        \gla am-\=a-ve-ra-m//
        \glb \root{LOVE}-\gsc{THEME}-Perf-Past-1\gsc{SG}//
        \glft `I loved'//
        \endgl
    \a \begingl
        \gla am-\=a-ve-r-\=o//
        \glb \root{LOVE}-\gsc{THEME}-Perf-Fut-1SG//
        \glft `I will have loved'//
        \endgl
\xe

V < Th < Perf < T.
\begin{itemize} \tightlist
    \item $\approx$ T > Perf > V.
    \item Like in English, except in morphology rather than syntax.
    \item Still get morphological conditioning, but we'll return to that later (allomorphy).
\end{itemize}

    \subsection{The Mirror Principle}
        \subsubsection{Preliminaries}
What's the order of the reciprocal and causative suffixes in Chiche\^{w}a \citep{alsina99}?
\ex[exno=3] \begingl
    \gla Al\=enje a-na-mény-\textbf{án}-\uline{its}-á mb\^{u}zi//
    \glb 2.hunters 2\gsc{s}-\gsc{PAST}-hit-\gsc{RECIP}-\gsc{CAUS}-\gsc{FinVwl} 10.goats//
    \glft `The hunters made the goats hit each other.'//
    \endgl
\xe
\ex[exno=4] \begingl
    \gla Al\=enje a-na-mény-\uline{éts}-\textbf{an}-a mb\^{u}zi//
    \glb 2.hunters \gsc{2S}-\gsc{PAST}-hit-\gsc{CAUS}-\gsc{RECIP}-\gsc{FV} 10.goats//
    \glft `The hunters made each other hit the goats.'//
    \endgl
\xe
The order is whatever it needs to be to get scope right; You go from the verb ``outwards''.

\bigskip
Here's another pair \citep{hymanmchombo92}:
\pex
    \a mang-\textbf{an}-\uline{its}\\
    tie-\gsc{RECIP}-\gsc{CAUS}\\
    `cause to tie each other'
    \a mang-\uline{its}-\textbf{an}\\
    tie-\gsc{CAUS}-\gsc{RECIP}\\
    `cause each other to tie'
\xe

What would this language (\emph{Chichewa'}) look like if it had prefixes instead of suffixes?
\pex
    \a its-an-mang
    \a an-its-mang
\xe

Because hierarchy is what matters (relative ordering), rather than linear order.

Another example: Bemba in \cite{baker85}, citing \cite{givon76}.
\pex[exno=49]
    \a \begingl
        \gla Naa-mon-an-ya Mwape na Mutumba//
        \glb 1s.\gsc{S}-\gsc{past}-see-\gsc{recip-caus} Mwape and Mutumba//
        \glft `I made Mwape and Mutumba see each other.'//
        \endgl
    \a \begingl
        \gla Mwape na Chilufya baa-mon-eshy-ana Mutumba//
        \glb Mwape and Chilufya 3p.\gsc{S}-see-\gsc{caus-recip} Mutumba//
        \glft `Mwape and Chilufya made each other see Mutumba.'//
        \endgl
\xe

        \subsubsection{Passivizing}
In Chichewa, the applicative can be used for instruments \citep{alsina99}:
\pex[exno=10] 
    \a \begingl
        \gla Ms\=odzi a-na-dúl-\textbf{ír}-a \textbf{nkhw\^{a}ngwa} uk\=onde//
        \glb 1.fisherman \gsc{1S}-\gsc{PAST}-cut-\gsc{APPL-FV} 9.axe 14.net//
        \glft `The fisherman cut the net with an axe.'//
        \endgl
\xe

Let's try to passivize: `The axe was used to cut the net (by the fisherman)'. The instrument will become the subject (we'll get back to that in argument structure). What will be the ordering of \gsc{Appl} and \gsc{Pass}?

\pex[exno=10]
    \a[label=b] \begingl
        \gla Nkhw\^{a}ngwa i-na-dúl-\textbf{ír}-\uline{idw}-á úk\=onde (ndí ms\=odzi)//
        \glb 9.axe 9\gsc{S}-\gsc{PAST}-cut-\gsc{APPL}-\gsc{PASS-FV} 14.net by 1.fisherman//
        \glft `The axe was used to cut the net (by the fisherman).'//
        \endgl
\xe

Schematic, ignoring the actual arguments:\\
\Tree
[.PassP
    [.Pass ]
    [.ApplP
        [.Appl ]
        \qroof{\emph{cut}}.vP
    ]
]    


Chi-Mwi:ni in \cite{baker85} from \cite{kisseberthabasheikh77}:
\pex[exno=56]   
    \a \begingl
        \gla Nu:ru Ø-chi-tes-ete chibu:ku//
        \glb Nuru \gsc{S-O}-bring-\gsc{asp} book//
        \glft `Nuru brought the book.'//
        \endgl
    \a \begingl
        \gla Nu:ru ø-m-tet-\textbf{el}-ele mwa:limu chibu:ku//
        \glb Nuru \gsc{S-O}-bring-\gsc{appl-asp} teacher book//
        \glft `Nuru brought the book to the teacher.'//
        \endgl
    \a \begingl
        \gla Mwa:limu ø-tet-\textbf{el}-el-\uline{a} chibu:ku na Nu:ru//
        \glb teacher \gsc{S}-bring-\gsc{appl-asp-pass} book by Nuru//
        \glft `The teacher was brought the book by Nuru.'//
        \endgl
\xe

Kinyarwanda in \cite{baker85} from \cite{kimenyi80}:
\pex[exno=57]
    \a \begingl
        \gla Umugabo a-ra-andik-a ibaruwa \textbf{n'i}-ikaramu//
        \glb man \gsc{S-pres}-write-\gsc{asp} letter with-pen//
        \glft `The man wrote [sic] the letter with the pen.'//
        \endgl
    \a \begingl
        \gla Umugabo a-ra-andik-\textbf{iish}-a ibaruwa ikaramu//
        \glb man \gsc{S-pres}-write-\gsc{instr-asp} letter pen//
        \glft `The man wrote the letter with the pen.'//
        \endgl
    \a \begingl
        \gla Ikaramu i-ra-andik-\textbf{iish}-\uline{w}-a ibaruwa n'umugabo//
        \glb pen \gsc{S-pres}-write-\gsc{instr-pass-asp} letter by-man//
        \glft `The pen was written-with [sic] the letter by the man.'//
        \endgl
    \a \begingl
        \gla Ibaruwa i-ra-andik-\textbf{iish}-\uline{w}-a ikaramu n'umugabo//
        \glb letter \gsc{S-pres}-write-\gsc{instr-pass-asp} pen by-man//
        \glft `The letter was written with the pen by the man.'//
        \endgl
\xe

        \subsubsection{Outside of Bantu}
Yupik \citep[43]{mithun99}:
\pex
    \a \begingl
        \gla ayag-ciq-\textbf{yugnarqe}-\uline{ni}-llru-u-q//
        \glb go-\gsc{FUT}-probably-claim-\gsc{PAST-INDIC.INTR}-3\gsc{SG}//
        \glft `He said he would probably go.'//
    \endgl
    \a \begingl
        \gla ayag-ciq-\uline{ni}-llru-\textbf{yugnarqe}-u-q//
        \glb go-\gsc{FUT}-claim-\gsc{PAST}-probably-\gsc{INDIC.INTR}-3\gsc{SG}//
        \glft `He probably said that he would go.'//
    \endgl
\xe
\ex \begingl
    \gla ayag-yug-umi-ite-qapiar-tu-a//
    \glb go-want-be.in.state-not-really-\gsc{INDIC.INTR}-1\gsc{SG}//
    \glft `I really don't want to go.'//
    \endgl
\xe
% (Why do we gloss these sometimes as words and sometimes as affixes? Because.)


Oji-Cree \citep{slavin05uot} in \cite{rice11}:
\pex[exno=11]
    \a \begingl
        \gla ishkwaa-niipaa-sookihpawn//
        \glb finish-at.night-be.snowing//
        \glft `It stopped snowing at night.' (does not snow at night anymore)//
    \endgl
    \a \begingl
        \gla nipaa-ishkwaa-sookihpwan//
        \glb at.night-finish-be.snowing//
        \glft `It stopped snowing at night.' (was snowing the whole day)//
    \endgl
\xe
\pex[exno=11']
    \a \begingl
        \gla kiimooci-kishahtapi-wiihsini//
        \glb secretly-fast-eat//
        \glft `He secretly eats fast.' (nobody knows that he eats fast)//
    \endgl
    \a \begingl
        \gla kishahtapi-kiimooci-wiihsini//
        \glb fast-secretly-eat//
        \glft `He eats secretly (nobody knows that he eats) and he does it fast.'//
    \endgl
\xe


Pulaar \citep{paster05}:
\pex
    \a \begingl
        \gla mi udd-id-it-ii baafe ɗe fof//
        \glb 1\gsc{SG} close-\gsc{COM-SEP-past} door Det all//
        \glft `I opened [sic] all the doors (in sequence).'//
        \endgl
    \a \begingl
        \gla mi udd-it-id-ii baafe ɗe fof//
        \glb 1\gsc{SG} close-\gsc{SEP-COM-past} door Det all//
        \glft `I opened all the doors (at once).'//
        \endgl
\xe
\pex
    \a \begingl
        \gla o jaŋŋg-in-it-ii kam//
        \glb 3\gsc{SG} learn-\gsc{CAUS-REP-past} 1\gsc{SG}//
        \glft `He taught me again.' (taught me before)//
        \endgl
    \a \begingl
        \gla o jaŋŋg-it-in-ii kam//
        \glb 3\gsc{SG} learn-\gsc{REP-CAUS-past} 1\gsc{SG}//
        \glft `‘He made me learn again.'//
        \endgl
\xe

    \subsection*{Interim summary}
\begin{itemize} \tightlist
    \item Rigid ordering for some (inflectional?) categories, e.g.~auxiliaries.
    \item Variable ordering depending on scope for some (inflectional?) categories.
    \item Always respect semantic and morphological scope.
\end{itemize}

Key references:
\cite{baker85} gets all the glory for coining the Mirror Principle but \cite{muysken81,muysken88} was there first. \cite{rice11} is an excellent overview of different factors in affix ordering, based in part on \cite{rice00}.


    \subsection{Longer chains}
Turkish \citep[368]{inkelasorgun98} 
\ex \begingl
	\glpreamble \emph{çekoslovakyalilaştiramayacaklarimizdanmiydiniz}//
	\gla çekoslovakya li laş tir ama yacak lar imiz dan mi ydi niz //
	\glb Czechoslovakia from become \gsc{CAUSE} unable Fut \gsc{PL} \gsc{1PL} \gsc{ABL} \gsc{INTERR} Past \gsc{2PL}//
	\glft `were you one of those whom we are not going to be able to turn into Czechoslovakians?'//
	\endgl
\xe

Japanese:
\ex \begingl
	\gla taro-ga kodomo-o \textbf{sodat-e-sase-rare-ta}//
	\glb Taro-\gsc{NOM} child-\gsc{ACC} rise-\gsc{CAUS}-\gsc{CAUS}-\gsc{PASS}-\gsc{PAST}//
	\glft `Taro was made to raise the child.'//
	\endgl
\xe

What would a structure for this look like?

\ex
\Tree
[.TP
	[.\emph{taro-ga} ]
	[.
		[.vP
			[.vP
				[.Ø ]
				[.
					[.vP
						[.\sout{\emph{taro-ga}} ]
						[.
							[.VP
								[.\emph{kodomo-o} ]		
								[.V \emph{sodat} ]
							]
							[.v\\{[\gsc{caus}]} \emph{e} ]
						]
					]
					[.v\\{[\gsc{caus}]} \emph{sase} ]
				]
			]
			[.v\\{[\gsc{pass}]} \emph{rare} ]
		]
		[.T \emph{-ta} ]
	]
]
\xe

Note how arguments (DPs, phrases) and suffixes live in harmony in the same tree.
% \emph{sodat-e} means `raise (a child or animal)' and \emph{sodat-u} is the inchoative `grow up'


    \subsection{Nouns}
Gender in Catalan, \cite{kramer16llc} from \cite{picallo91}:
\pex
    \a \begingl
        \gla el gos-ø//
        \glb the.\gsc{M} dog.\gsc{M}//
        \endgl
    \a \begingl
        \gla els goss-o-s//
        \glb the-\gsc{PL} dog-\gsc{M}-\gsc{PL}//
        \endgl
\xe
\pex
    \a \begingl
        \gla la goss-a//
        \glb the.F dog-F//
        \endgl
    \a \begingl
        \gla les goss-e-s//
        \glb the.\gsc{F.PL} dog-\gsc{F-PL}//
        \endgl
\xe

Schematic tree:
\ex \label{tree:pl-catalan}
\Tree
[.
    [.
        [.\emph{goss} ]
        [.\gsc{[F]} \emph{e} ]
    ]
    [.\gsc{[PL]} \emph{s} ]
]
\xe

Yupik \citep[43]{mithun99}
\pex
    \a yug-\textbf{pag}-\uline{cuar}\\
    person-big-little\\
    `little giant'
    \a yug-\uline{cuar}-\textbf{pag}\\
    person-little-big\\
    `big midget'
\xe

English:
\ex glob-al-iz-ation \label{ex:globalization}
\xe
\ex novel-iz-ation-s
\xe

Let's draw a structure for~(\ref{ex:globalization}):
\ex
\Tree
[.[N]
    [.[V]
        [.[A]
            [.[N] \emph{glob} ]
            [.[A] \emph{al} ]
        ]
        [.[V] \emph{ize} ]
    ]
    [.[N] \emph{ation} ]
]
\xe

Is this morphology or syntax? Does it matter, and if so, in what ways?

    \subsection{Building structure}
Quick reminder: what is the evidence for hierarchical structure within words, as opposed to linear structure?

        \subsubsection{Ordering}
\pex \emph{unhelpful}:
    \a {[}un-[help-ful]]
    \a \ljudge{*} {[[}un-help]-ful] 
\xe
\pex \emph{unpredictable}:
    \a {[}un-[predict-able]]
    \a \ljudge{*} {[[}un-predict]-able] 
\xe

\pex But also \emph{unhappier} (a bracketing paradox, because -\emph{er} normally attaches to smaller prosodic bases):
    \a \ljudge{*} {[}un-[happi-er]]
    \a {[[}un-happi]-er] 
\xe

See also \cite{libben03springer} for processing aspects and \cite{osekimarantz20} for modelling.

\emph{[[en-joy]-ment]} vs *\emph{[en-joy]ful}.

        \subsubsection{Ambiguity}
Where there's structure, there can be structural ambiguity.

\pex
    \a \emph{unlockable, unfoldable}, \dots{} \hfill \citep{dealmedialibben05,pollatseketal10}
    \a Compounds like \emph{kitchen towel rack}
\xe

        \subsubsection{A formal implementation}
Schematic trees for:
\ex  
    \Tree
    [.\emph{glob-al-iz-ation-s}
        [.[N]
            [.[V]
                [.[A]
                    [.[N] \emph{glob} ]
                    [.[A] \emph{al} ]
                ]
                [.[V] \emph{ize} ]
            ]
            [.[N] \emph{ation} ]
        ]
        [.[\gsc{PL}] \emph{s} ]
    ]
\xe

\ex 
    \Tree
    [.\emph{redd-en-ed}
        [.[V]
            [.[A] \emph{red} ]
            [.[V] \emph{en} ]
        ]
        [.[\gsc{PAST}] \emph{ed} ]
    ]
\xe

% \ex solid-ifi-ed
% \xe

Are we putting everything in the same tree or is there some point in which we want to take the ``morphological tree'' and put it within a ``syntactic tree''?

It's hard to talk about building up words (or phrases, or sentences) without a concrete theory of morphology, a concrete theory of syntax, and empirical domains in which to test them. I'm also of the opinion that it's extremely hard to talk about processing or computational aspects of morphology (or syntax) without a formal theory. So we'll flesh out one kind of theory that blurs the distinction between morphology and syntax somewhat, mostly in order to give ourselves a concrete starting point.

\bigskip
Back to technical concerns: what about bound roots?
\pex
    \a \emph{eternity} \\
    \Tree
        [.n
            [.\root{etern} ]
            [.n \emph{ity}\\{[N]} ]
        ]
    \a \emph{eternal} \\
    \Tree
        [.a
            [.\root{etern} ]
            [.a \emph{al}\\{[A]} ]
        ]
\xe

\bigskip
With that in place, what's to stop us from always positing a root? What would that look like for the noun \emph{globe}?
\ex
\Tree
[.n
    [.\root{\gsc{globe}} ]
    [.n ]
]
\xe

Nothing really, though we would need grounds to decide which of the following two derivations to prefer:
\ex \emph{global}\\
    a.~\Tree
    [.a
        [.n
            [.\root{\gsc{globe}} ]
            [.n ]
        ]
        [.a\\\emph{al} ]
    ]
    \hspace{5em}
    b.~\Tree
    [.a
        [.\root{\gsc{GLOBE}} ]
        [.a\\\emph{al} ]
    ]
\xe

\bigskip
Next, we would expect to find languages that have these kinds of roots which must be \emph{categorized} as a noun, verb or adjective first.

$\Rightarrow$ \cite{arad03}!


    \subsection{Back to inflection and derivation}

Let's recap once more the differences between inflection and derivation. We want to see whether our way of building structure helps explain these.

\begin{tabular}{ll}
Inflection & 
Derivation \\\hline
Forced by syntactic context & 
Not forced by syntactic context\\ 
Productive in all lexical categories &  
Limited productivity \\
More regular (morphologically? Semantically?) & 
Less regular \\
Does not change category &
Sometimes changes category \\ 
Does not change meaning (compositional)  &
Usually changes meaning \\
Doesn't create new stems & 
Creates new stems \\
Farther away from the base & 
Closer to the base \\
\end{tabular}

\bigskip
Starting from the last line, we now have an \emph{explanation} for why derivation is closer to the root: it's scope! It's scope all the way down!

\bigskip
We can also look at meaning change again. Do we get meaning change with \emph{every} derivational affix? Does our theory explain this?
\pex
    \a globalization
    \a novelization
\xe

\bigskip
What about non-compositional meaning?




\section{More factors affecting affix ordering}
    \subsection{Syntactic vs semantic ordering}

\cite{rice00,rice11}: syntactic role (subj/obj) vs semantic role (Agent/Theme) in Athapaskan.

Some data from Koyukon \citep{thompson96ijal,jettejones00}:
\pex[exno=2]
    \a Transitive subject / Agent\\
    dee-\textbf{n}-’oyh\\
    `you (sg) will handle object'
    \a Intransitive subject / Agent\\
    da-\textbf{a}-l-’onh\\
    `you (sg) are in position'
    \a Transitive object / Theme\\
    \textbf{ne}-henee-l-’aanh\\
    `they are looking at you (sg)'
\xe

\begin{itemize} \tightlist
    \item Hypothesis A: Theme first, Agent second.
    \item Hypothesis B: Object first, Subject second.
\end{itemize}

So Rice went to \cite{thompson96ijal} to test the predictions:
\pex[exno=2]
    \a[label=d] Passive subject / Theme\\
    eetegh-\textbf{ee}-l-dzes\\
    `You will be hit once.'
\xe

Hypothesis A is ruled out. The ordering is syntactic, rather than purely semantic.

    \subsection{Phonological factors}

We'll now examine phonological factors influencing affix ordering. Why?
\begin{itemize} \tightlist
    \item Interesting in its own right because it's a thing that exists.
    \item We'll want to be on the lookout for cases in which the phonological factors conflict with the structural factors.
\end{itemize}

        \subsubsection{Kashaya infixation}
Phonological placement of the pluractional affix in Kashaya (Pomoan) from \cite{buckley00}. 
\pex
    \a dahqoƭol + ta $\rightarrow$ dahqoƭol-ta-\\
    `fail (to do)'
    \a diƭ'an + ta $\rightarrow$ diƭ'an-ta-\\
    `bruise by dropping’
\xe
\pex 
    \a bilaqʰam + ta $\rightarrow$ bilaqʰa-ta-m\\
    `feed'
    \a sima:q + ta $\rightarrow$ sima-ta-q\\
    `go to sleep'
\xe

If the root ends in a coronal, the pluractional -\emph{ta} is a suffix, else infix.

But this is about infix and base, not two affixes. See \cite{kalin20inf} on infixes as affixes.

        \subsubsection{Affixes in Gombe Fula}
Ordering in Gombe Fula \citep{arnott70}, as discussed by \cite{paster05,paster06phd} and recapped in \cite{rice11}.
\ex \begin{tabular}{llll}
    (-ɗ & DENominative & fur-ɗ-a & `be grey') \\
    -t & REVersive & taar-t-a & `untie'\\
    -t & REPetitive & soor-t-o & `sell again'\\
    -t & REFlexive & ndaar-t-o & `look at oneself'\\
    -t & RETaliative & jal-t-o & `laugh at \dots{} in turn'\\
    -t & INTensive & yan-t-a & `fall heavily'\\
    -d & ASSociative & nast-id-a & `enter together'\\
    -d & COMprehensive & janng-id-a & `read, learn all \dots{}'\\
    -n & CAUsative & woy-n-a & `cause to cry'\\
    -r & MODal & ɓe mah-ir-i Îi & `they built them with'\\
    -r & LOCative & ’o ’yiw-r-ii & `he came from'\\
    \end{tabular}
\xe

Examples of these combining \citep[235]{paster06phd}:
\pex
    \a \begingl
        \gla ’o maɓɓ-\textbf{it}-\textbf{id}-ii jolɗe fuu//
        \glb 3sg close-REV-COM-past doors all//
        \glft `he opened all the doors' (\citealt{arnott70}: 367)//
        \endgl
    \a \begingl
        \gla ’o yam-ɗ-\textbf{it}-\textbf{in}-\textbf{ir}-ii mo lekki gokki kesi//
        \glb 3sg_i healthy-DEN-REV-CAU-MOD-past 3sg_j medicine other new//
        \glft `he_i cured him_j with some new medicine' (368)//
        \endgl
    \a \begingl
        \gla war-\textbf{t}-\textbf{ir}-//
        \glb come-REV-MOD-//
        \glft `bring back' (367)//
        \endgl
    \a \begingl
        \gla ’o maɓɓ-\textbf{it}-\textbf{ir}-ii yolnde hakkiil//
        \glb 3sg close-REV-MOD-past door slowly//
        \glft `he opened the door slowly' (367)//
        \endgl
    \a \begingl
        \gla no njooɗ-\textbf{od}-\textbf{or}-too mi ’e maɓɓe//
        \glb how sit-ASS-MOD-rel.fut 1sg with 3pl//
        \glft `how shall I sit/live with them?' (367)//
        \endgl
\xe

\cite{arnott70} claims that affixes in Gombe Fula are ordered according to their phonological identity. But maybe they... arnott!

\cite{paster06phd} identifies a phonological generalization. What is it?

This is sonority ordering!

\bigskip
But what about reverse orderings? Well TDNR \emph{is} the baseline scopal ordering. Yet we do find different orders and they're what we'd expect.

[[speak with] again]], not [[speak again] with], which is what TD would've given us:
\pex \a \begingl
    \gla mi wol-\textbf{d}-\textbf{it}-at-aa ’e maɓɓe//
    \glb 1sg speak-COM-REP-fut-neg with 3pl//
    \glft `I won’t speak with them again'//
        \endgl
    \a \begingl
        \gla ’o nyaam-n-id-ii ɗi//
        \glb 3sg eat-CAU-COM-past 3pl//
        \glft `He fed them all’, `He made it so that they all eat'//
    \endgl
\xe

        \subsubsection{More phonological ordering}

% (21a) sonority but irrelevant to semantic scope because never co-occur! (no data, check paper)

Navajo, \cite{youngmorgan87} in \cite{rice11}: \emph{j} unspecified subject and \emph{ʔ} unspecified object.

\pex
    \a n-\textbf{ʔ}-\textbf{ji}-ɬ-go'\\
        `unspecified pokes around (something)'
    \a dá-da-\textbf{zh}-\textbf{ʔ}-di-nil\\
        `unspecified plugs up (holes)'
\xe
Simplified:
\ex V ʔ j C $\rightarrow$ V zh ʔ C
\xe

The two affixes do not take scope over one another!

Western Cherokee in \cite{manova21wiley}, from \cite{foley80}: when do we get metathesis?
\ex
    \begin{tabular}{l|ll|ll}
     & Counterfactive & & Cislocative & \\
    \gsc{1SG} & yi-ji-nega & $\rightarrow$ yijinega & wi-ji-nega & $\rightarrow$ wijinega\\
    \gsc{3SG} & yi-ga-nega & $\rightarrow$ yiganega & wi-ga-nega & $\rightarrow$ wiganega\\
    \gsc{2SG} & \textbf{y}i-hi-nega & $\rightarrow$ h\textbf{y}inega &  \textbf{wi}-hi-nega & $\rightarrow$ h\textbf{w}inega\\
    \end{tabular}
\xe

Meathesis in 2\gsc{SG} but no scope. Cherokee doesn't have clusters of a glide plus a non-vocalic segment.

I don't know what \emph{nega} means because we just take data from communities!

    \subsection{Exceptions to the Mirror Principle?}
Prosodic size of affix in Slave, \cite{rice89} in \cite{rice11}. 
\pex[exno=27]
    \a \begin{tabular}{lll}
        \multicolumn{2}{l}{Non-possessed} & Possessed\\
        ʔah &  `snowshoes'  &  ʔah-\'ɛʔ\\
        -zha    & diminutive    & \\
        ʔah-zha & `small snoeshows, women's snowshoes' & ʔah-\'ɛ-zha\\
        \end{tabular}
    \a \begin{tabular}{lll}
        \multicolumn{2}{l}{Non-possessed} & Possessed\\
        beh &  `knife' & -beh-\'ɛʔ \\
        -cho & augmentative & \\
        beh-cho & `large knife', `hunting knife’ & -beh-\'ɛ-cho
        \end{tabular}
\xe

The prosodically smaller possessive affix appears closer to the root, even though the diminutive or augmentative should be closer to the root. Definitely merits a closer look!

\bigskip
Next, \cite{myler17oup} from \cite{hyman03}. Proto-Bantu had:
\ex[exno=7] \begingl
    \gla Caus- Appl- Rec- Caus- Pass//
    \glb -ic- -id- -an- -į- -u//
    \endgl
\xe

\pex[exno=8] Spirnatization in Nyakusa \citep[105]{myler17oup}:
    \a sat- ‘be in pain’ $\rightarrow$ sas-į ‘give pain’
    \a ag- ‘run out’ $\rightarrow$ as-į ‘make run out’
    \a gel ‘measure’ $\rightarrow$ ges-į ‘try’
    \a tup ‘become thick’ $\rightarrow$ tuf-į ‘thicken’
    \a olob ‘become rich’ $\rightarrow$ olof-į ‘make rich’
    \a sok ‘go out’ $\rightarrow$ sos-į ‘take out’
\xe

Reciprocal \emph{-an-} always precedes causative \emph{-į-}. So does the intervening \emph{-an-} block spirantization?

\cite{myler17oup} now citing a 2000 handout of Hyman's:
\pex
    \a sob- `get lost (intr.)'
    \a sof-į `to lose (tr.)'
\xe
\pex
    \a sob-an-į `get each other lost' (causativized reciprocal)
    \a sof-an-į `to lose each other' (reciprocalized causative)
\xe
So the order starts scopal and then you have movement.

\bigskip
From a 2011 talk by Kiparsky, the \emph{ruki} rule in Sanskrit applies with some prefixes like the prepositional-like elements here, which retroflex the stem-initial consonant:
\pex[exno=31]
    \a abhi + siñc- $\rightarrow$ abhi-\d{s}iñc- `pour on'
    \a adhi + sthā- $\rightarrow$ adhi-\d{s}\d{t}hā- `stand on'
\xe

But the augment (inflectional) \emph{-a-} counterbleeds ruki:
\pex[exno=32]
    \a abhi + siñc- $\rightarrow$ abhi-\d{s}iñc- $\rightarrow$ abhi-a-\d{s}iñc-a-m `poured on'
    \a adhi + sthā- $\rightarrow$ adhi-\d{s}\d{t}hā-  $\rightarrow$ adhi-a-\d{s}\d{t}ā-n `stood on' 
\xe
This makes sense if it's a high inflectional element and the first P then raises above it.

    \subsection{Summary}
\begin{itemize} \tightlist
    \item Semantic scope is read off of (morpho-)syntactic structure.
    \item Trees work well, regardless of whether we're dealing with morphemes or words, phrases or clauses.
    \item By and large, phonological factors do not override syntactic factors.
\end{itemize}

    \subsection{Even more?}
Huave (Kim, Noyer).

Hap(lo)ology.

Computational angle, e.g.~\cite{ryan10}.

\cite{mueller20}

        \subsubsection{Templates}
CARP (Hyman, Paster)

Nimboran/Inkelas/Noyer

\citet[194]{rice11} has a nice quote from Hyman which is pro-templates, and mentions processing and social factors as additional possibilities.


\section{Allomorphy}

What is allomorphy? What changes, and based on what? Let's find some examples.

    \subsection{Preliminaries}
        \subsubsection{Phonological conditioning}
Our \textbf{suffix} depends on the \textbf{phonology} of the stem:
\pex \label{ex:past-en}
    \a \emph{grade}[əd] 
 	\a \emph{jam}[d] 
 	\a  \emph{jump}[t] 
\xe

Why?

A formalization (does it capture the reason?):

\ex  T[Past] \lra $\begin{cases} 
	\emph{\text{əd}} & / \text{[+cor --cont --son]} ~ \trace \\
	\text{\emph{t}} & /  \text{[--voice]} ~ \trace \\
	\text{\emph{d}} & \\
	\end{cases}$
\xe

Well it works, but if the generalization is about coronals and voicing, we want that to drive the formalization or at least be clear in it.

Is this phonology or morphology?
\begin{itemize} \tightlist
    \item There's a phonological logic at play.
    \item If it's phonology, you'd expect similar patterns elsewhere. If we look at the \emph{-s/-z-əz} suffix, we do see the same beahvior in the present tense, the plural and the possessive.
    \item But remember the major lesson from \cite{wug}: you can't rely solely on phonology to decide what the variant is after sonorants, so there's at least \emph{some} morphology in the morphophonlogy.
\end{itemize}

\bigskip
Our \textbf{determiner} depends on the \textbf{phonology} of the noun:
\pex \label{ex:en-indef}
    \a   \emph{a dog}		 
 	\a  \emph{an apple} 
\xe

Why?

Formally (does it capture the reason?):
\ex
  D[\textminus{}def] \lra $\begin{cases} 
	\emph{\text{ə}} & / \trace~ \#\text{C} \\
	\emph{\text{ən}} & / \trace ~ \#\text{V} \\
	\end{cases}$ 
\xe

This is much harder to capture in purely phonological forms: we don't insert /n/ to break of hiatus in English.

        \subsubsection{Interlude: Allomorphy as optimization}
In~(\ref{ex:past-en}) and~(\ref{ex:en-indef}) the alternations make phonotactic sense. In fact, the past tense alternation could be treated in the phonology proper (with phonological rules), and see \cite{gouskovaetal15} and \cite{pak16} on the indefinite article in English.

Does allomorphy always make phonological sense?

\bigskip
Data from \cite{embick10}:
\pex[exno=2]
    \a Korean nominative: optimizing\\
        \begin{tabular}{llll}
        Allomorph   &   Environment & Example & Gloss\\ \hline
        -i  & / C \trace{} & pap-i & `cooked rice'\\
        -ka & / V \trace{} & ai-ka  & `child'\\
        \end{tabular}
    \bigskip
    
    \a Seri passive suffix \citep{marlettstemberger83}: not great but better than pC \\
        \begin{tabular}{llll}
        Allomorph   &   Environment & Example & Gloss\\\hline
        p-  & / \trace{} V & -p-eʃi & `be defeated'\\
        a:ʔ-    & elsewhere & -a:ʔ-kaʃni & `be bitten'\\
        \end{tabular}
    \bigskip
    
    \a Haitian Creole definite: non-optimizing \\
        \begin{tabular}{llll}
        Allomorph   &   Environment & Example & Gloss\\\hline
        -la & / C \trace{} & liv-la & `book'\\
        -a & / V \trace{} & tu-a & `hole'\\
        \end{tabular}
\xe
\bigskip

Allomorphy is not necessarily optimizing. 

        \subsubsection{Syntactic conditioning}
\emph{Suppletion}: our \textbf{root} depends on the \textbf{syntactic features}.
\pex
    \a \emph{go} (today) 
 	\a  \emph{went} (yesterday) 
\xe

\ex
$\begin{cases} 
	\text{\emph{go}} & / \trace~ \text{[\gsc{PRS}]} \\
	\text{\emph{went}} & / \trace ~ \text{[\gsc{PAST}]} \\
	\end{cases}$
\xe
 
Also with adjectives:
\pex
    \a \emph{good}		 
 	\a  \emph{better} 
 	\a  \emph{best} 
\xe

\ex  \gsc{GOOD} \lra $\begin{cases} 
	\text{\emph{good}} & / \trace~ \text{[\gsc{NORM}]} \\
	\text{\emph{better}} & / \trace ~ \text{[\gsc{CMPR}]} \\
	\text{\emph{best}} & / \trace ~ \text{[\gsc{SPRL}]} \\
	\end{cases}$
\xe 

Why? (diachronic reasons)

        \subsubsection{What's possible?}

We need an inventory of relevant elements. Which of these do we consider?
\pex
    \a Syntactic features
    \a Phonological features
    \a Lexical idiosyncrasies (diacritics?)
    \a Semantics (e.g.~the comparative case -- is it syntax, semantics or both?)
    \a Something like CxG Constructions
    \a \dots{}
\xe

Where is allomorphy? Who is allomorphy? I'll do you one better: Why is allomorphy?

\bigskip
We will:
\begin{itemize} \tightlist
    \item Establish a structural hierarchy based on scope.
    \item Establish the generalization.
    \item See how it fits with cases where we need slightly more fine-grained syntax.
\end{itemize}

    \subsection{Syntactic: Inward sensitivity}
        \subsubsection{English}
PL in English is sensitive to the root (we're talking about the suffix now, not root suppletion!):

\pex[exno=4] \gsc{pl} \lra{}
    \a -en / \{ \root{\gsc{ox}}, \dots \} \trace{}
    \a -Ø / \{ \root{\gsc{sheep}}, \root{\gsc{deer}}, \dots \}
    \a \dots
    \a -s
\xe

What's the relevant structure? We already proposed something in~(\ref{tree:pl-catalan}):
\ex \Tree
[.
    [.
        [.{\emph{goss}\\`dog'} ]
        [.\gsc{[F]} \emph{e} ]
    ]
    [.\gsc{[PL]} \emph{s} ]
]
\xe

So if we want we can add the categorizer and find the direction of allomorph triggering:
\ex 
\Tree
[.
    [.
        [.\root{\gsc{root}} ]
        [.(n/Gen) ]
    ]
    [.\gsc{[PL]} ]
]
\xe

        \subsubsection{Latin}
Latin recap:
\ex
\Tree
[.TP
    [.\emph{The cake} ]
    [.
        [.T\\\emph{will} ]
        [.PerfP/AspP
            [.Perf\\\emph{have} ]
            [.ProgP
                [.Prog\\\emph{been} ]
                [.PassP
                    [.Pass\\\emph{being} ]
                    [.VP\\\emph{eaten} ]
                ]
            ]
        ]
    ]
]
\xe

Latin:
\pex
    \a \begingl
        \gla am-\=a-ve-ra-m//
        \glb \root{LOVE}-\gsc{THEME}-Perf-Past-1\gsc{SG}//
        \glft `I loved'//
        \endgl
    \a \begingl
        \gla am-\=a-ve-r-\=o//
        \glb \root{LOVE}-\gsc{THEME}-Perf-Fut-1SG//
        \glft `I will have loved'//
        \endgl
\xe

Linear order (and scopal order): V < Th < Perf < T.

Representing~(\lastx b) as a hieararchical structure, rather than linear (with some extra assumptions that aren't important now):
\ex \Tree
[.TP
    [. ]
    [.
        [.PerfP
            [. ]
            [.
                [.VoiceP
                    [.(DP) ]
                    [.
                        [.vP
                            [.\root{root}\\\emph{am}- ]
                            [.v
                                [.v ]
                                [.\gsc{TH}\\-\emph{\=a}- ]
                            ]
                        ]
                        [.\gray{(Voice)} ]
                    ]
                ]
                [.Perf\\-\emph{ve}- ]
            ]
        ]
        [.T 
            [.T{[Fut]}\\-\emph{r}- ]
            [.Agr\\-\emph{\=o} ]
        ]
    ]
]
\xe

Latin T is conditioned by Perf:

\pex \label{ex:perf-agr-past}
    \a \begingl
        \gla am-\=a-\underline{ba}-\textbf{m}//
        \glb \root{am}-\gsc{TH}-\underline{Past}-\textbf{\gsc{1SG}}//
        \glft `I loved'//
    \endgl

    \a \begingl
        \gla am-\=a-\fbox{ve}-\underline{ra}-\textbf{m}//
        \glb \root{am}-\gsc{TH}-\fbox{Perf}-\underline{Past}-\textbf{\gsc{1SG}}//
        \glft `I had loved'//
    \endgl
\xe

If our theory is about hierarchy and nodes, rather than features or values as such, then we'd predict that the interaction between Perf and T might hold even for other values of T. That is, this isn't about the present or past perfect, it's about the nodes Perf and T. At least in Latin this is the case:

\pex \label{ex:perf-agr-fut}
    \a \begingl
        \gla am-\=a-\underline{b}-\textbf{\=o}//
        \glb \root{am}-\gsc{TH}-\underline{Fut}-\textbf{\gsc{1SG}}//
        \glft `I will love'//
    \endgl

    \a \begingl
        \gla am-\=a-\fbox{ve}-\underline{r}-\textbf{\=o}//
        \glb \root{am}-\gsc{TH}-\fbox{Perf}-\underline{Fut}-\textbf{\gsc{1SG}}//
        \glft `I will have loved'//
    \endgl
\xe

        \subsubsection{More}
\cite{rolle21wiley}:
\pex Outward-conditioning by a morphosyntactic feature (inward sensitivity)
    \a Bulgarian \citep{gribanovaharizanov17oup} - inner gender and number features condition form of outer [DEFINITE] suffix
    \a Moro \citep{jenkssande17} – inner feature [PROPER] (on names and kinship terms) conditions overt exponence of outer [ACCUSATIVE] case
    \a Modern Standard Arabic \citep{winchester17nels} – an inner gender feature [-FEM] conditions a form of an outer number feature [+AUGMENTED]
    \a Bengali \citep{banerjee20lsa} – inner feature [PERFECT] conditions form of outer [NEGATIVE]
\xe

\textbf{Observation:} inward sensitivity to syntactic features (and/or diacritics).


    \subsection{Syntax: Outward sensitivity}
What would outward sensitivity in syntactically-conditioned allomorphy look like?
\begin{itemize} \tightlist
    \item Affixes conditioning roots (root suppletion).
    \item Affixes conditioning affixes that essentially come before them.
\end{itemize}


        \subsubsection{Suppletion in Hiaki and Korean}
Some verbs in Hiaki supplete by number \citep{harley14thlia}:
\ex
    \begin{tabular}{l >{\em}l >{\em}l l}
    & \gsc{SG} & \gsc{PL} & \\\hline
    a.& vuite & tenne & `run' \\
    b.& siika & saka & `go'\\
    c.& weama & rehte & `wander'\\
    d.& kivake & kiime & `enter'\\
    e.& vo'e & to'e & `lie'\\
    f.& weye & kaate & `walk'\\
    g.& mea & sua & `kill' (\gsc{sg}$\sim$\gsc{pl} object)\\
    \end{tabular}
\xe

The root is sensitive to the value of a higher T/Agr/number node.
\bigskip

Honorification is affixal in Korean \citep{choiharley19}:
\pex
    \a \begingl
    \gla Halapeci-kkeyse cip-ey ka*(-si)-ess-ta//
    \glb grandfather-NOM.HON home-to go-HON-PST-DECL//
    \glft `Grandfather went home.'//
    \endgl
    \a \begingl
    \gla Halapeci-kkeyse cip-i khu-si-ta//
    \glb grandfather-NOM.HON house-NOM big-HON-DECL//
    \glft `Grandfather’s house is big.'//
    \endgl
\xe

\ex Hon \lra{} \emph{si}-
\xe

\bigskip
But a few verbs supplete:
\ex \label{ex:korean-sleep-ellipsis}
    \begingl
    \gla Halapeci-kkeyse-nun yekise, aki-nun cekise ca-ss-ta//
    \glb grandfather-NOM.HON-TOP here baby-TOP there sleep-PST-DECL//
    \glft `Grandfather slept here and the baby slept there.' \trailingcitation{[(39)]}//
    \endgl
\xe

\ex \begingl
    \gla Halapeci-kkeyse pang-eyse cwumwusi-ess-ta//
    \glb grandfather-NOM.HON room-at sleep.HON-PST-DECL//
    \glft `Grandfather slept in the room.'//
    \endgl
\xe

\pex
    \a ca- ‘sleep’ $\sim$ cwumwusi- ‘sleep.HON’
    \a mek- ‘eat’ $\sim$ capswusi- ‘eat.HON’
    \a iss- ‘exist’ $\sim$ kyeysi- ‘exist.HON’
\xe

\pex \root{\gsc{sleep}} \lra{}
    \a cwumwusi- / \trace{} Hon
    \a ca-
\xe

\bigskip
How do we know that these are really two forms of the same root (lemma), rather than verbs from two separate paradigms?

Because idioms are perserved under honorifics \citep[1343]{choiharley19}:
\ex \begingl
    \gla kkamakwi koki-lul mek/capswusi-ta//
    \glb crow meat-\gsc{ACC} eat/eat.\gsc{HON}-\gsc{DECL}//
    \glft `to forget' (lit.~'eat crow meat')//
    \endgl
\xe

And also ellipsis as in~(\ref{ex:korean-sleep-ellipsis}).

        \subsubsection{Multiple affixes in Georgian}
Applicatives in Georgian with the verb \emph{rbina} `run': \hfill \citep{hewitt95,bonetharbour12}
\pex 
    \a mo-\textbf{m}-\uline{i}-rbina `\textbf{I} had run here'
    \a mo-\textbf{gv}-\uline{i}-rbina `\textbf{we} had run here'
    \a mo-\textbf{g}-\uline{i}-rbina(t) `\textbf{you} all had run here'\\
		\hdashrule[2pt][c]{0.5\linewidth}{1pt}{1.5mm}
	\a mo-\textbf{Ø}-\uline{u}-rbina(t) `\textbf{he/she/they} had run here'
\xe
\ex \gsc{appl} \lra{} $\begin{cases}
    u & / [3~[~\trace{}~]] \\
    i
    \end{cases}$
\xe
\bigskip

Itelmen \citep{bobaljik00}: class morpheme depends on Agr. \citet[10--11]{gouskovabobaljik20cup}:

\ex /Ø-tɸɬ-z-čŋ-in/ [tɸsčŋin] `you are bringing it'
\xe

\includegraphics[width=0.5\linewidth]{figs/itelmen.png}

\pex Subject
    \a S:1sg \lra{} t-
    \a S:2 \lra{} \zero{}
\xe
\pex Root
    \a \root{\gsc{bring}} \lra{} tɸɬ
    \a \root{\gsc{embrace}} \lra{} taβol
\xe
\pex Inflection class
    \a cl \lra{} -čŋ / ]2 ] O:3, S:2sg, realis
    \a cl \lra{} -xk / ]2 ] O:[+participant]
    \a cl , -k(i) / ]2
\xe
\pex Object
    \a O:3sg \lra{} -in / ] S:2sg, realis
    \a O:3sg \lra{} -nen / ]S:3
    \a O:3sg \lra{} -čen / ]
\xe

\textbf{Observation:} outward sensitivity to syntactic features (and/or diacritics).

        \subsubsection{Summary}
Allomorphy is sensitive to morphosyntactic triggers that can be either lower (inner) or higher (outer) than the exponent.

Questions (for another time):
    \begin{enumerate} \tightlist
        \item Can any feature be a trigger?
        \item How far away can triggers be? (Locality)
    \end{enumerate}


    \subsection{Phonology: Inward sensitivity}
        \subsubsection{Arabic}
Moroccan Arabic (and other dialects), \cite{rolle21wiley} from \cite{mascaro07} from \cite{harrell62}:
\ex
    \begin{tabular}{l>{\em}ll>{\em}ll}
    a.& xtʕa-h & `his error'  & ktab-u & `his book'\\
    b.& ʃafu-h & `they saw him' & ʃaf-u & `he saw him'\\
    c.& mʕa-h & `with him' & menn-u & `from him'\\
    \end{tabular}
\xe

\bigskip
\pex \gsc{[POSS.3SG.M]} \lra{}
    \a \emph{h} / V \trace{}
    \a \emph{u}
\xe
Issues? Again, doesn't explain the phonotactic logic behind the alternation (historically -hu). Recall also the Korean nominative (optimizing), but the Haitian definite article too (anti-optimizing).
% \ex
%         \begin{tabular}{llll}
%         % Allomorph   &   Environment & Example & Gloss\\ \hline
%         -i  & / C \trace{} & pap-i & `cooked rice'\\
%         -ka & / V \trace{} & ai-ka  & `child'\\
%         \end{tabular}
% \xe

\cite{paster06phd} has more examples of C/V conditioning.

        \subsubsection{Other}
Here's a different natural class, in Tahitian \citep{lazardpeltzer00}:
\ex
    \begin{tabular}{l>{\em}ll>{\em}ll}
    a.& 'amu & `eat' & \textbf{fa'a}-'amu & `make eat'\\
    b.& rave & `do, make' & \textbf{fa'a}-rave & `make make'\\
    c.& tai'o & `read' & \textbf{fa'a}-tai'o & `make read'\\\hline
    d.& mana'o & `think' & \textbf{ha'a}-mana'o & `remember'\\
    e.& fiu & `grow tired' & \textbf{ha'a}-fiu & `be bored'\\
    f.& veve & `be poor' & \textbf{ha'a}-veve & `impoverish'\\
    \end{tabular}
\xe

Here it's labials that condition the form of the affix. They're adjacent on the C tier but not on the segmental tier.

\bigskip
Chaha \citep{jjmcc83}:
\ex Palatalization in the imperative\\
    \begin{tabular}{lll}
    2\gsc{SG.M} & 2\gsc{SG.F} & gloss \\\hline
    nomæd & nomædʲ & `love' \\
    noqo\d{t} & noqo\d{t}ʲ & `kick'\\
    goræz & goræzʲ & `be old'\\
    \end{tabular}
\xe
\ex Labialization in the perfective\\
    \begin{tabular}{lll}
    wihtout object & with object & gloss \\\hline
    qænæf & qænæfʷ & `knock down'\\
    nækæb & nækæbʷ & `find'\\\hline
    nækæs & nækʷæs & `bite' \\
    kæfæt & kæfʷæt & `open' \\\hline
    qæ\d{t}ær & qʷæ\d{t}ær & `kill'\\
    mæsær & mʷæsær & `seem'\\
    \end{tabular}
\xe

Not just the segmental tier but the featural tier! And there's even spreading (in other examples).


\bigskip
Tone-conditioned allomorphy (possessive?) in Mixtepec Mixtex \citep{pastermbeamdeazcona05,paster06phd}: -yù after low tone, otherwise floating L:
\pex Stem ends in H:
    \a vílú + Ⓛ → vílúù 'my cat'
    \a nàmá + Ⓛ → nàmáà 'my soap'
\xe
\pex Stem ends in M:
    \a tzàákū + Ⓛ → tzàákūù 'my coral'
    \a kwà'ā + Ⓛ → kwà'āà 'my sister'
\xe
\pex Stem ends in L:
    \a tūtù + yù → tūtù yù 'my paper'
    \a chá'à + yù → chá'à yù 'I am short'
\xe

\bigskip
Tzeltal perfective depends on the syllable count of the stem \citep{walshdickey99,paster06phd}:
\pex
    \a j-al-\textbf{oh} `he has told something'
    \a s-mah-\textbf{oh} `he has hit something' 
    \a s-kutʃ-\textbf{oh} `she has carried it'
\xe
\pex
    \a s-tikun-\textbf{ɛh} `he has sent something' 
    \a s-mak'lin-\textbf{ɛh} `he has fed someone' 
    \a s-kutʃ-laj-\textbf{ɛh} `she was carrying it repeatedly'
\xe

\bigskip
\cite{rolle21wiley} from \cite{austin81,deakpilbara08}: in Thalanyji, case marking on nouns is sensitive to whether the noun is disyllabic or not. Case marking on pronouns and demonstratives is invariant across both. Except for DAT, whose form is different for each of the three.

See also Western Armenian \citep{dolatian22gjgl}.

\bigskip
\textbf{Observation:} inward sensitivity to phonological material (might be phonologically adjacent in a specific way, like on the C tier).


    \subsection{Phonology: Outward sensitivity}
What would this look like?

Affix conditions root:
\begin{enumerate}
	\item Root: \emph{pata} `dog'.
	\item With C-initial suffix: \emph{pata-\textbf{k}i} `the dog', \emph{pata-\textbf{m}u} `my dog'.
	\item With V-initial suffix: \emph{\uline{der}-\textbf{a}} `dogs', \emph{\uline{der}-\textbf{e}} `your dog'.
\end{enumerate}
No language like this has been reported! When we find something similar, we also find alternative analyses \citep{adgeretal03tops,kalin20jomo,kastner21li,kiparsky21tlr,dolatian22gjgl}.

    \subsection{Theory building}
        \subsubsection{Restrictive theories}
\paragraph{Phonology} Generative phonology has gone through different formalisms.

What's wrong with this?
\ex /v/ $\rightarrow$ [f] / \trace{} [t]
\xe

\begin{enumerate} \tightlist
    \item You lose the generalization that this is about voicing.
    \begin{itemize}
        \item There might be a general devoicing/assimilation rule in the language.
    \end{itemize}
    \item You could write anything!
    \ex /g/ $\rightarrow$ [u] / \# \trace{} [n]
    \xe
\end{enumerate}

So rule-based phonology concentrated on features from early on \citep{spe}:
\ex {[}+voiced] $\rightarrow$ [--voiced] / \trace{} [--voiced]
\xe

\bigskip
What about that? Well we could still write the following rule using the same formalism:
\ex {[}--voiced] $\rightarrow$ [+voiced] / \trace{} [--voiced]
\xe

And while cases of dissimilation like~(\lastx) are attested, you still lose the generalization that you're either assimilating or dissimilating. So we have alpha-notation:
\ex {[}$\pm$voiced] $\rightarrow$ [$\alpha$ voiced] / \trace{} [$\alpha$ voiced]
\xe

All explicit theories try to account for what's possible and what should be impossible. The strongest ones find ways of building that into the architecture, making strong testable claims.

\paragraph{Syntax} Why was the advent of X-bar theory so cool?

Phrase structure grammar:
\pex
    \a S $\rightarrow$ NP VP
    \a NP $\rightarrow$ D (Adj) NP
    \a NP $\rightarrow$ (Adj) N
    \a VP $\rightarrow$ V NP
    \a VP $\rightarrow$ V NP PP
    \a VP $\rightarrow$ \dots{}
    \a PP $\rightarrow$ P NP\\
		\hdashrule[2pt][c]{0.21\linewidth}{1pt}{1.5mm}
    \a V $\rightarrow$ N D VP?
    \a NP $\rightarrow$ \emph{dog} Adv S?
\xe

X-bar theory essentially said:
\pex
    \a XP $\rightarrow$ YP X'
    \a X' $\rightarrow$ X ZP
    \a Figure out what the lexical content of X, Y and Z is in your language.
\xe
So:
\begin{enumerate} \tightlist
    \item Explains parallels in internal structure of NP, VP, PP, etc.
    \item Predicts head-finality more or less correctly in head-final languages.
\end{enumerate}

\paragraph{Semantics} Variables must be bound in their domain. This is why e.g.~donkey sentences are so fascinating to formal semanticists.

\paragraph{Summary} These aren't the only relevant subfields, of course. But:
    \begin{itemize} \tightlist
        \item In each case you're setting up the architecture to tell you both what is and isn't possible.
        \item And then you can think about the origin of that architecture (be it cognitive, historical, etc).
    \end{itemize}

        \subsubsection{A theory of allomorphy}
Let's remind ourselves of what we need to derive (the \emph{explananda}):
\begin{itemize} \tightlist
    \item General distinction between syntactic features and phonological features.
    \item Asymmetry: low material can't see high phonological material (but can see high syntactic material).
    \item Productivity: productive rules exist but so does suppletion.
\end{itemize}

\textbf{What should the architecture look like?} Here are some relevant questions.
\paragraph{Hierarchical relations.} We're assuming a syntactic/morphological/morphosyntactic structure, rather than a linear one (or paradigms, or anything else). Why?
\begin{itemize} \tightlist
    \item Affix order reflects word order.
    \item Words have internal structure so why not.
    \item Inward/outward is not necessarily the same as right/left (though we haven't seen that as such, but we'd expect interactions with syntax).
    \item Uniform module so why not.
\end{itemize}

\paragraph{Late Insertion.} Now let's think about what suppletion teaches us.
\begin{itemize} \tightlist
    \item \emph{Go/went} aren't like \emph{dog$\sim$cat}: one set of syn-sem info is associated with two different forms.
    \item How do you know which form to choose? Depending on the syn and phono information around you.
    \item But when do you know? You can only do that once you have that information.
    \item Lexicalist theories assume that the syntax combines full words. Would that work?
        \begin{itemize} \tightlist
            \item No, because how would you know what word to put at the base of your tree if you don't have the rest of it with the conditioning information yet!
            \item So we need a node to be a bundle of syntax, and only pick the form once you have other nodes with the relevant info around it.
        \end{itemize}
\end{itemize}

\paragraph{Cyclic spell-out.} We still need to understand the contrast between phonological and syntactic conditioning.
\begin{itemize} \tightlist
    \item Where do syntactic and phonological features live in the structure?
    \item How does that relate to Late Insertion?
    \item If you start at the bottom and insert one step at a time, then you can only see the phonological form that's below, not above.
    \item So the different assumptions, when combined, create an architecture which \emph{derives} the generalization.
\end{itemize}

\paragraph{Summary.} This is more or less contemporary Distributed Morphology, as developed most clearly by \cite{embick10}. There are still many questions we haven't explored, such as what the cycles are like, what the locality relations are, and how else you might deal with any of these issues.

\paragraph{Caveat.} Other theories do have ways of dealing with e.g.~suppletion. Some of the even-stronger evidence for Late Insertion theories comes for patterns of syncretism. See for example Ruth Kramer's recent/upcoming chapter in the CUP DM handbook.


    \subsection{Locality}
        \subsubsection{Prelude: Reminder of \cite{arad03}}
\pex Forms derived from \root{sgr} (after \citealt[746]{arad03}): \label{ex:sgr}
    \a \emph{seger} `closure' (N)
    \a \emph{sograim} `parentheses' (N)
    \a \emph{sgira} `closing' (N)
    \a \emph{sagur} `closed' (A)
    \a \emph{misgeret} `frame' (V)
    \a \emph{sagar} `closed' (V)
    \a \emph{histager} `cocooned himself' (V)
\xe

\bigskip
The verb \emph{misger} `framed' has one consonant ``too many''. \cite{arad03,arad05}: two things persist once you derive the noun.
\begin{itemize} \tightlist
    \item Phonology: The /m/ persists since the verb is derived from the noun \emph{misgeret}, rather than from the root \root{sgr}. 
    \item Semantics: The meaning of `frame' persists in all derived (denominal) forms.
\end{itemize}

\pex
    \a \cmark{} \root{sgr} $\rightarrow$ \emph{misgeret}$_{\text{N}}$ $\rightarrow$ \emph{misger}$_{\text{V}}$
    \a \xmark{} \root{sgr} $\rightarrow$ \emph{misger}$_{\text{V}}$
\xe

\ex \label{ex:kip-trees1}
    a. \Tree 
    [.{v\\\emph{misger}} v [.{n\\\emph{misgeret}} n \root{sgr} ] ]
    \hskip 2cm 
    b. \Tree  [.{n\\\emph{seger}} n \root{sgr} ]
\xe

        \subsubsection{Latin}
Discussion from \cite{embick10,kastnerzu17}.

Latin tree:
\ex \label{tree:latin}
%\begin{small}
\scalebox{0.95}{
a. \Tree
[.TP
    [. ]
    [
        [.T ]
        [.(PerfP)
            [.(Perf) ]
            [.VoiceP
                [.(DP) ]
                [
                    [.Voice ]
                    [.vP
                        [.\root{root} ]
                        [.v ]
                    ]
                ]
            ]
        ]
    ]
]
{\Large $\Rightarrow$}
b. \Tree
[.TP
    [. ]
    [.
        [.PerfP
            [. ]
            [.
                [.VoiceP
                    [.(DP) ]
                    [.
                        [.vP
                            [.\root{root}\\\emph{am}- ]
                            [.v
                                [.v ]
                                [.\gsc{TH}\\-\emph{\=a}- ]
                            ]
                        ]
                        [.\gray{(Voice)} ]
                    ]
                ]
                [.Perf\\-\emph{ve}- ]
            ]
        ]
        [.T 
            [.T{[Fut]}\\-\emph{r}- ]
            [.Agr\\-\emph{\=o} ]
        ]
    ]
]
}
\xe

\paragraph{Perf Conditions T}
The past suffix is usually -\emph{ba}, but is -\emph{(e)ra} when Perf appears.
\pex
    \a  \begingl
        \gla am-\=a-\textbf{ba}-t//
        \glb \root{am}-\gsc{TH}-\textbf{Past}-3\gsc{SG}//
        \glft `he/she loved' \hfill (Default -\emph{ba}-)//
        \endgl
    \a \begingl
        \gla am-\=a-\underline{ve}-\textbf{ra}-t//
        \glb \root{am}-\gsc{TH}-\underline{Perf}-\textbf{Past}-\gsc{3SG}//
        \glft `he/she had loved' \hfill (Special -\emph{ra}-)//
        \endgl
\xe

\paragraph{Root Conditions Perf, Theme Blocks the Conditioning}\label{sec:latin:analysis:perfroot}
What conditions what?
\pex\label{ex:perf-root}
    \a 
        \begingl
        \gla am-\=a-\textbf{vi}-mus//
        \glb \root{\gsc{LOVE}}-\gsc{TH}-\textbf{Perf}-\gsc{1PL}//
        \glft `we have loved'// % (Default -\emph{v}/\emph{vi}-)//
    \endgl
    
    \a 
        \begingl
        \gla scrip-\textbf{si}-mus//
        \glb \root{\gsc{write}}-\textbf{Perf}-\gsc{1PL}//
        \glft `we have written'// % (Special -\emph{si}-)//
    \endgl
    
    \a 
        \begingl
        \gla v\=en-\textbf{i}-mus//
        \glb \root{\gsc{come}}-\textbf{Perf}-\gsc{1PL}//
        \glft `we have come'// % (Special -\emph{i}-)//
    \endgl
\xe

\citet[72]{embick10}: Perf conditioned by the root, if the two are adjacent. The theme vowel needs to be null. This pattern is consistent with a locality-based approach if the root and Perf are linearly adjacent.

You might know that Latin has conjugation classes, but this is really about the root rather than class: different allomorphs of Perf appear in \emph{men}-\textbf{\emph{u}}-\emph{\={\i}} `I have warned', \emph{aug}-\textbf{\emph{s}}-\emph{\={\i}} `I have increased' and \emph{str\={\i}d}-\textbf{\emph{i}}-\emph{\={\i}} `I have whistled', all second conjugation (Class II).

\paragraph{Perf Conditions Agr, T Blocks the Conditioning}
What conditions what?
\pex \label{ex:perf-agr-pres}
    \a \begingl
        \gla am-\textbf{\=o}//
        \glb \root{am}-\gsc{TH}.\textbf{\gsc{1SG}}//
        \glft `I love'//
    \endgl

    \a \begingl
        \gla am-\=a-\fbox{v}-\textbf{\={\i}}//
        \glb \root{am}-\gsc{TH}-\fbox{Perf}-\textbf{\gsc{1SG}}//
        \glft `I have loved'//
    \endgl
\xe
\pex 
    \a \begingl
        \gla am-\=a-s//
        \glb \root{love}-\gsc{TH}-\gsc{2SG}//
        \glft `you love'//
    \endgl

    \a \begingl
        \gla am-\=a-\fbox{v}-\textbf{isti}//
        \glb \root{love}-\gsc{TH}-\fbox{Perf}-\gsc{2SG}//
        \glft `you have loved'//
    \endgl
\xe

% The \gsc{2SG} ending is usually -\emph{s}, but in the present perfect it is -\emph{ist\={\i}}. The person/number ending of the present is conditioned by linearly adjacent Perf, whereas an overtly realized T blocks the conditioning. With a null T in the present, Perf is linearly adjacent to Agr and can condition contextual allomorphy. When T is overt, as in the past or the future, the default endings appear. 
Agr conditioned by Perf (inward-looking).

\bigskip
But we also saw T intervening between Perf and Agr: the 1\gsc{SG} ending for a Class I root like \root{am} `love' is consistent\textemdash \emph{m} in the past and \emph{o} in the future (depending on the value of T).

\pex \label{ex:perf-agr-past2}
    \a \begingl
        \gla am-\=a-\underline{ba}-\textbf{m}//
        \glb \root{am}-\gsc{TH}-\underline{Past}-\textbf{\gsc{1SG}}//
        \glft `I loved'//
    \endgl

    \a \begingl
        \gla am-\=a-\fbox{ve}-\underline{ra}-\textbf{m}//
        \glb \root{am}-\gsc{TH}-\fbox{Perf}-\underline{Past}-\textbf{\gsc{1SG}}//
        \glft `I had loved'//
    \endgl
\xe

\pex \label{ex:perf-agr-fut2}
    \a \begingl
        \gla am-\=a-\underline{b}-\textbf{\=o}//
        \glb \root{am}-\gsc{TH}-\underline{Fut}-\textbf{\gsc{1SG}}//
        \glft `I will love'//
    \endgl

    \a \begingl
        \gla am-\=a-\fbox{ve}-\underline{r}-\textbf{\=o}//
        \glb \root{am}-\gsc{TH}-\fbox{Perf}-\underline{Fut}-\textbf{\gsc{1SG}}//
        \glft `I will have loved'//
    \endgl
\xe

\bigskip
T is closer to Agr, so it gets to condition it. And/or: Agr is conditioned by Perf, but T intervenes.


See also \citet[71]{embick10}, \cite{carstairsmccarthy01} and~\cite{adgeretal03tops}.

\paragraph{Theme Conditions T, Perf Blocks the Conditioning}
Traditional grammars: the person endings in the perfect do not change from conjugation to conjugation. What does this mean? Why does it come about?

\bigskip
First, endings usually change from conjugation to conjugation (associated with the theme vowel):
\pex 
    \a \begingl
        \gla am-\underline{\=a}-\textbf{b}-\=o//
        \glb \root{am}-\underline{\gsc{TH}}-Fut-\gsc{1SG}//
        \glft `I will love' (Class I)//
        \endgl
    \a \begingl
        \gla pet-\underline{a}-\textbf{m}//
        \glb \root{pet}-\underline{\gsc{TH}}-Fut.\gsc{1SG}//
        \glft `I will seek' (Class III)//
        \endgl
\xe

\bigskip
What happens in the perfect?
\pex
    \a \begingl
        \gla am-\underline{\=a}-\fbox{ve}-\textbf{r}-o//
        \glb \root{am}-\underline{\gsc{TH}}-\fbox{Perf}-\textbf{Fut}-\gsc{1SG}//
        \glft `I will have loved' (Class I)//
        \endgl
    \a \begingl
        \gla pet-\underline{\=\i}-\fbox{ve}-\textbf{r}-o//
        \glb \root{pet}-\underline{\gsc{TH}}-\fbox{Perf}-\textbf{Fut}-\gsc{1SG}//
        \glft `I will have sought' (Class III)//
        \endgl
\xe

\bigskip
Perf intervenes between TH and T/Agr (and chooses its own allomorph of T).

\paragraph{Summary:} Our theory of allomorphy needs to make reference to linear adjacency, which reflects structural adjacency.

        \subsubsection{Nonconcatenative locality: Hebrew}
We'll now test this theory in a domain where we wouldn't expect linear adjacency: nonconcatenative morphology \citep{kastner18nllt,kastner20ogs}.

We need some background and assumptions first.

\paragraph{Background} Semitic verbs come in different ``templates''.

\ex\label{ex:naive-ktb} Some forms for \root{ktb}, generally associated with writing.\\
	\begin{tabular}{lllll}
	& Template & Verb & Gloss & Note\\\hline
	a. & \tkal & katav & `wrote' & unmarked/transitive\\
	b. & \tnif & nixtav & `was written' & anticausative of \tkal~(\ref{ex:naive-ktb}a)\\
	c. & \thif & hextiv & `dictated' & causative of \tkal~(\ref{ex:naive-ktb}a)\\
	d. & \thuf & huxtav & `was dictated' & passive of \thif~(\ref{ex:naive-ktb}c)
	\end{tabular}
\xe
We'll assume that the templates are encoded in Voice, because they reflect argument structure.

\paragraph{Roots}
Some roots change the form of the stem. Assume that the stem vowels are part of the phonology of the template, so are in Voice.
\ex\label{ex:tkal-reg}Some regular roots in \tkal:\\
 \begin{tabular}{llc|ll}
	& \multicolumn{2}{c|}{Root} & Past \gsc{3SG.M} \emph{XaYaZ} & Future \gsc{3SG.M} \emph{jiXYoZ}\\\hline
	a. & `write' & \root{ktb}& katav & jixtov \\
	b. & `wash' & \root{ʃtf}& ʃataf & jiʃtof \\
	c. & `break' & \root{ʃbr} & ʃavar & jiʃbor \\
\end{tabular}
\xe

\ex\label{ex:tkal-irreg}Some irregular roots in {\tkal}:\\
 \begin{tabular}{c|llc|ll|ll}
	Class & & \multicolumn{2}{c|}{Root} & \multicolumn{2}{c|}{Past \gsc{3SG.M}} & \multicolumn{2}{c}{Future \gsc{3SG.M}}\\\hline
	\multirow{2}{*}{/j/-final \root{XYj}}
		& a. & `happen' & \root{\dgs{k}rj}& kar\uline{a}&(*kara\uline{j}) & jikr\uline{e} & (*jikroj) \\
		& b. & `want' & \root{r\texttslig{}j}& ra\texttslig{}\uline{a}&(*ra\texttslig{}a\uline{j}) & jir\texttslig{}\uline{e} & (*jir\texttslig{}oj) \\\hline
	\multirow{1}{*}{/ʔ/-final \root{XYʔ}}
		& c. & `freeze' & \root{\dgs{k}pʔ}& kaf\uline{a}&(*kafa\uline{ʔ}) & jikp\uline{a} & (*jikpoʔ) \\\hline
	\multirow{3}{*}{/n/-initial \root{nYZ}}
		& d. & `fall' & \root{npl}& nafal& & j\uline{ip}ol & (*jinpol) \\
		& e. & `give' & \root{ntn}& natan& & j\uline{i}t\uline{e}n & (*jinton) \\
		& f. & `avenge' & \root{n\dgs{k}m}& nakam& & ji\uline{n}kom &  \\
\end{tabular}
\xe

Similar effects arise in other templates, for instance in~\tpie~in~(\nextx).
\ex\label{ex:tpie-irreg}A regular and irregular root in \tpie:\\
 \begin{tabular}{c|llc|l|l}
	Class & & \multicolumn{2}{c|}{Root} & Past \gsc{3SG.M} & Future \gsc{3SG.M}\\\hline
	Regular \root{XYZ}
		& a. & `complicate' & \root{sbx}& sibex & jesabex  \\
	Doubled \root{XYY}
		& b. & `spin' & \root{svv}& s\uline{o}vev & jes\uline{ov}ev  \\
\end{tabular}
\xe

Voice is conditioned by Root:
\ex
\Tree
    [.TP
        [.\tikz{\node (TAgr) {T+Agr};} ]
        [
            [.\tikz{\node (Voice) {Voice};} ]
            [
                [.v\\{(covert)} ]
                [.\tikz{\node (Root) {\root{root}};} ]
            ]
        ]
    ]
\begin{tikzpicture}[overlay]
    \draw[dotted,thick,->] (Root) .. controls +(south:1) and +(south:2) .. (Voice);
\end{tikzpicture}
\xe

\paragraph{T and Voice} Let's look at the stem vowels in a few paradigms (two different templates).
\ex \label{ex:1st2nd}
	\begin{tabular}{l|ll}
		 & \multicolumn{2}{c}{\emph{hegdil} `enlarged'}\\
		 & \gsc{SG} & \gsc{PL}\\	\hline
		\red{1} & hegd\textbf{\green{a}}l-\blue{t}i & hegd\textbf{\green{a}}l-\blue{n}u\\
		\red{2\gsc{M}} & hegd\textbf{\green{a}}l-\blue{t}a & hegd\textbf{\green{a}}l-\blue{t}em\\
		\red{2\gsc{F}} & hegd\textbf{\green{a}}l-\blue{t} & hegd\textbf{\green{a}}l-\blue{t}em\\\hline
		3\gsc{M} & hegd\textbf{i}l-\zero{} & hegd\textbf{i}l-u\\
		3\gsc{F} & hegd\textbf{i}l-a & hegd\textbf{i}l-u
	\end{tabular} \hfil
	\begin{tabular}{l|ll}
	 & \multicolumn{2}{c}{\emph{gidel} `grew'}\\
	 & \gsc{SG} & \gsc{PL}\\	\hline
		\red{1} & gid\textbf{\green{a}}l-\blue{t}i & gid\textbf{\green{a}}l-\blue{n}u\\
		\red{2\gsc{M}} & gid\textbf{\green{a}}l-\blue{t}a & gid\textbf{\green{a}}l-\blue{t}em\\
		\red{2\gsc{F}} & gid\textbf{\green{a}}l-\blue{t} & gid\textbf{\green{a}}l-\blue{t}em\\\hline
		3\gsc{M} & gid\textbf{e}l-\zero{} & gid\sout{\textbf{\gray{e}}}l-u\\
		3\gsc{F} & gid\sout{\textbf{\gray{e}}}l-a & gid\sout{\textbf{\gray{e}}}l-u
	\end{tabular}
\xe

\bigskip
The stem vowel depends on the subject's person value.

\pex \label{tree:locality1}
\Tree
    [.TP
        [.\tikz{\node (TAgr) {T+Agr};} ]
        [
            [.\tikz{\node (Voice) {Voice};} ]
            [
                [.v\\(covert) ]
                [.\tikz{\node (Root) {\root{root}};} ]
            ]
        ]
    ]
    \begin{tikzpicture}[overlay]
    %\draw[semithick,->] (TAgr) to[out=south west,in=south west] (Voice);
    \draw[dotted,thick,->] (Root) .. controls +(south:1) and +(south:2) .. (Voice);
    %\draw[dotted,thick,->] (TAgr) to[out=225,in=270] (Voice);
    \draw[dotted,thick,->] (TAgr) .. controls +(south west:1) and +(south west:1) .. (Voice);
    \end{tikzpicture}
\xe

\paragraph{Prediction: Root and T+Agr}
What might we predict next about the interaction of Root, Voice and T+Agr?

\begin{enumerate} \tightlist
    \item Root and Voice: linearly adjacent. Already seen that Root conditions Voice.
    \item Voice and T+Agr: linearly adjacent. Different agreement affixes depending on the template. This is correct.
    \item Root and T+Agr: overt Voice intervenes. The same agreement affixes are used regardless of root class. This is correct.
\end{enumerate}

\ex\label{tree:locality2}
\Tree
    [.TP
        [.\tikz{\node (TAgr) {T+Agr};} ]
        [
            [.\tikz{\node (Voice) {Voice};} ]
            [
                [.{v\\(covert)} ]
                [.\tikz{\node (Root) {\root{root}};} ]
            ]
        ]
    ]
    \begin{tikzpicture}[overlay]
    \draw[dotted,thick,->] (TAgr) .. controls +(north west:2) and +(north east:2) .. (TAgr);
    \draw[dotted,thick,->] (Voice) .. controls +(south:1) and +(south:1) .. (TAgr);
    \draw[dotted,thick,->,red] (Root) .. controls +(south:2) and +(south west:2) .. node{\large $\times$}(TAgr);
	\end{tikzpicture}
\xe

\bigskip
\paragraph{Prediction: Passives}
What would happen if we now have something between Voice and T+Agr? Enter the passive!

The difference we saw in~(\ref{ex:1st2nd}) no longer exists, here for \emph{biʃel} `cooked':
\ex
	\begin{tabular}{ll|ll}
	& Template & Past \gsc{3SG.M} & Past \gsc{1SG} \\\hline
	a.& \tpie~(active)  & bi\textipa{S}\underline{\'e}l & bi\textipa{S}\underline{\'a}l-ti \\
	b.& \tpua~(passive) & b\textbf{u}\textipa{S}\textbf{\underline{\'a}}l & b\textbf{u}\textipa{S}\textbf{\underline{\'a}}l-ti \\
	\end{tabular}
\xe

\bigskip
Tense doesn't matter either: there might be a difference in vowels between past and future in the active,~(\nextx a), but not in the passive,~(\nextx b).
\ex 
	\begin{tabular}{ll|ll}
	& Template & Past \gsc{3SG.M} & Future \gsc{3SG.M} \\\hline
	a.& \tpie~(active)  & b\underline{i}\textipa{S}\'el & je-v\underline{a}\textipa{S}\'el \\
	b.& \tpua~(passive) & b\textbf{\underline{u}}\textipa{S}\textbf{\'a}l & je-v\textbf{\underline{u}}\textipa{S}\textbf{\'a}l \\
	\end{tabular}
\xe

\bigskip
\textbf{Generalization:} The stem vowels are always \emph{-u-a-} in the passive.

Full paradigms:
\pex\label{table:pass-vowels}
%\a Past tense for passive \emph{pukad} `was commanded' and \emph{hufkad} `was deposited':\\
\a Past of passive \emph{gudal} `was raised' and \emph{hugdal} `was enlarged':\\
\begin{tabular}{l||ll|ll}
 & \multicolumn{2}{c}{\tpua~\root{gdl}}	& \multicolumn{2}{c}{\thuf~\root{gdl}}\\
 & \gsc{SG} & \gsc{PL}	& \gsc{SG} & \gsc{PL}\\\hline
1 & g\textbf{u}d\textbf{\'a}l-ti & g\textbf{u}d\textbf{\'a}l-nu		& h\textbf{u}gd\textbf{\'a}l-ti & h\textbf{u}gd\textbf{\'a}l-nu\\
2M & g\textbf{u}d\textbf{\'a}l-ta & g\textbf{u}d\textbf{\'a}l-tem	& h\textbf{u}gd\textbf{\'a}l-ta & h\textbf{u}gd\textbf{\'a}l-tem\\
2F & g\textbf{u}d\textbf{\'a}l-t & g\textbf{u}d\textbf{\'a}l-tem	& h\textbf{u}gd\textbf{\'a}l-t & h\textbf{u}gd\textbf{\'a}l-tem\\
3M & g\textbf{u}d\textbf{\'a}l & g\textbf{u}d\del{\textbf{\'a}}l-{\'u}	& h\textbf{u}gd\textbf{\'a}l & h\textbf{u}gd\del{\textbf{\'a}}el-{\'u}\\
3F & g\textbf{u}d\del{\textbf{\'a}}l-{\'a} & g\textbf{u}d\del{\textbf{\'a}}l-{\'u}	& h\textbf{u}gd\del{\textbf{\'a}}el-{\'a} & h\textbf{u}gd\del{\textbf{\'a}}el-{\'u}
\end{tabular}

\a Future of passive \emph{jegudal} `will be raised' and \emph{jugdal} `will be enlarged':\\
%\a Future tense for passive \emph{jefukad} `will be commanded' and \emph{jufkad} `will be deposited':\\
\begin{tabular}{l||ll|ll}
 & \multicolumn{2}{c}{\tpua~\root{gdl}}	& \multicolumn{2}{c}{\thuf~\root{gdl}}\\
 & \gsc{SG} & \gsc{PL}	& \gsc{SG} & \gsc{PL}\\\hline
1 & j-e-g\textbf{u}d\textbf{\'a}l & n-e-g\textbf{u}d\textbf{\'a}l		& j-\textbf{u}gd\textbf{\'a}l & n-\textbf{u}gd\textbf{\'a}l\\
2M & t-e-g\textbf{u}d\textbf{\'a}l & t-e-g\textbf{u}d\del{\textbf{\'a}}l-{\'u}	& t-\textbf{u}gd\textbf{\'a}l & t-\textbf{u}gd\del{\textbf{\'a}}el-{\'u}\\
2F & t-e-g\textbf{u}d\del{\textbf{\'a}}l-{\'i} & t-e-g\textbf{u}d\del{\textbf{\'a}}l-{\'u}	& t-\textbf{u}gd\del{\textbf{\'a}}el-{\'i} & t-\textbf{u}gd\del{\textbf{\'a}}el-{\'u}\\
3M & j-e-g\textbf{u}d\textbf{\'a}l & j-e-g\textbf{u}d\del{\textbf{\'a}}l-{\'u}	& j-\textbf{u}gd\textbf{\'a}l & j-\textbf{u}gd\del{\textbf{\'a}}el-{\'u}\\
3F & t-e-g\textbf{u}d\textbf{\'a}l & j-e-g\textbf{u}d\del{\textbf{\'a}}l-{\'u}	& t-\textbf{u}gd\textbf{\'a}l & j-\textbf{u}gd\del{\textbf{\'a}}el-{\'u}
\end{tabular}
\xe

\bigskip
And this is exactly what we'd expect:
\ex\label{tree:locality3}
\Tree
    [.TP
        [.\tikz{\node (TAgr) {T+Agr};} ]
        [
	        [.\textbf{Pass}\\{\tikz{\node (Pass) {\textbf{\emph{u}}};}} ]
	        [
	            [.Voice\\{\tikz{\node (Voice) {\emph{\textcolor{red}{he,a}}};}} ]
	            [
	                [.v\\{(covert)} ]
	                [.\tikz{\node (Root) {\root{root}};} ]
	            ]
	        ]
        ]
    ]
    \begin{tikzpicture}[overlay]
    \draw[dotted,thick,->] (TAgr) .. controls +(north west:2) and +(north east:2) .. (TAgr);
    \draw[dotted,thick,->,red] (TAgr) .. controls +(south west:3) and +(south west:2) .. node{\large $\times$}(Voice);
    \draw[dotted,thick,->] (Voice) .. controls +(south:1) and +(south:2) .. node{\large $\times$}(TAgr);
    \draw[dotted,thick,->] (Root) .. controls +(south:2) and +(south west:3) .. node{\large $\times$}(TAgr);
    \end{tikzpicture}
\xe





    \subsubsection{Cyclicity: French prepositions and determiners}
\paragraph{D-N cliticization.}
\citet[87]{embick10}: adjoin D to vowel-initial nouns when adjacent.
\pex
    \a \emph{le chat} `the cat' (M)
    \a \emph{la mère} `the mother' (F)
\xe
\pex
    \a \emph{l'arbre} `the tree' (M), *\emph{le arbre}
    \a \emph{l'abeille} `the bee' (F), *\emph{la abeille}
\xe

\ex
\Tree
    [.DP
        [. ]
        [.
            [.D ]
            [.NP [.{N\\\emph{C-/V-}} ] ]
        ]
    ]
\xe
So you need some allomorphy-like rule for D and N (which are almost always adjacent), which cares about the phonology of inner N.

\paragraph{P-D fusion.} Some prepositions ``fuse'' with the definite article: \emph{de} `of' and \emph{à} `to' in the masculine, plus \emph{à} in the feminine plural:
\ex
    \begin{tabular}{l>{\em}l>{\em}ll}
        & \normalfont{``Fused''} & \normalfont{Separate} & gloss\\\hline
    \multirow{3}{*}{Masc} & du chat  & *de le chat   & `of the cat'\\
        & au chat   & *à le chat    & `to the cat'\\
        & aux chats & *à les chats  & `to the cats'\\\hline
    \multirow{3}{*}{Fem} & * & de la mère    & `of the mother'\\
        & * & à la mère & `to the mother'\\
        & aux mères & *à les mères & `to the mothers'\\
    \end{tabular}
\xe

\ex
\Tree
[.PP
    [.P ]
    [.DP
        [. ]
        [.
            [.D ]
            [.NP [.{N\\\emph{C-/V-}} ] ]
        ]
    ]
]
\xe
So we need a rule which combines P and D, depending on their syntactic features/diacritics.

\paragraph{Cyclicity.} What would be a situation in which either rule could apply?

With a V-initial noun, e.g.~\emph{de} + \emph{le} + \emph{arbre} `of the tree'? One rule applies to P+D and the other to D+N. What are the predictions in each case?

\ex
\Tree
[.PP
    [.{P\\\emph{de}} ]
    [.DP
        [. ]
        [.
            [.{D\\\emph{le}} ]
            [.NP [.{N\\\emph{arbre}} ] ]
        ]
    ]
]
\xe

\bigskip
No fusion:
\pex \a \emph{de l'arbre}
    \a \ljudge{*} \emph{du arbre}
\xe

\bigskip
Why no fusion? Because you first adjoin D to N. This makes sense if \textbf{the derivation proceeds cyclically, from the bottom up}.

% stopped at p. 92 of his book




    \subsection{Discussion}
\paragraph{Summary.} We've ended up with a cyclic, modular, bottom-up, local theory of allomorphy. Along the way we've assumed that syntax and morphology \{are, can be\} the same module, and Late Insertion of phonological material.

\paragraph{Meaning.} Recall: what did \cite{arad03} conclude about how meaning is computed across derivations based on Hebrew?

\paragraph{Locality conditions across domains.} Combining Arad's proposal with the locality conditions on allomorphy, \cite{marantz13} argues for similar locality conditions for allomorphy and meaning. This is an architectural point: the structure (syntax/morphology) imposes constraints on how each module (phonology, morphology) gets its own information.

\bigskip
Combining the unpredictable meaning with special allomorphy for English nouns \citep{embick10}:
\ex[exno=31]
    \begin{tabular}{>{\em}l>{\em}l}
    refus-al & refus-ing\\
    marri-age & marry-ing\\
    destruct-ion & destruct-ing\\\hline
    dance-\zero{} & danc-ing\\
    break-Ø & break-ing\\\hline
    curios-ity & curious-ness\\
    \end{tabular}
\xe

\bigskip
One possible (Arad-style) analysis:
\ex
    a. \Tree 
    [.{n\\\emph{refusal}}
        [.\root{\gsc{refuse}} ]
        [.{n\\\emph{-al}} ]
    ]
    \hspace{2cm}
    b. \Tree
    [.{n\\\emph{refusing}} 
        [.v
            [.\root{\gsc{refuse}} ]
            [.{v\\\zero{}} ]
        ]
        [.{n\\\emph{-ing}} ]
    ]
\xe

See \cite{kastner19cup} for an overview of this last bit.

\paragraph{Locality conditions.} Locality in allomorphy might be weaker than strict adjacency: \cite{kastnermoskal18}, \cite{merchant15li,christopoulospetrosino17}, Nanosyntax, \cite{sandeetal20}.


    \subsection{Further reading}

\cite{gouskovabobaljik20cup}, \cite{bonetharbour12}, \cite{bobaljik00}, \cite{adgeretal03tops}


    \subsection*{Bonus: Hungarian (material incomplete)}
Hungarian (also \citealt{tabachnik21dp}):

Plural is -(V)\emph{k} (depends on the stem, plus there's vowel harmony):
\pex 
    \a csont-ok\\
    `bones'
    \a fog-ak\\
    `teeth'
\xe

Possession is marked on the possessed item:
\pex
    \a \begingl
        \gla az (\H{o}) csont -ja -Ø//
        \glb the her bone -\gsc{poss} -\gsc{3sg}//
        \glft `her bone'//
    \endgl
    \a \begingl
        \gla az (\H{o}) fog -a -Ø//
        \glb the her tooth -\gsc{poss} -\gsc{3sg}//
        \glft `her tooth'//
    \endgl
\xe

\ex[exno=62] Hungarian plural possessive \citep[165]{carstairs87}\\
    \begin{tabular}{llllll}
    & \gsc{sg}  & \gsc{sg}, \gsc{1sg.poss}    & \gsc{pl}  & \gsc{pl}, \gsc{1sg.poss} & Gloss\\\hline
    a.& ruha & ruhá-m & ruhá-k & ruha-ái-m & `(my) dress(es)'\\
    b.& kalap & kalap-op & kalap-ok & kalap-jai-m & `(my) hat(s)'\\
    c.& ház & ház-am & ház-ak & ház-ai-m & `(my) house(s)'\\
    \end{tabular}
\xe

From \cite{rolle21wiley}:
\ex \cite{carstairsmccarthy01}\\
    \begin{tabular}{llll}
    & \gsc{SG} & \gsc{PL} & \\
    a.& dal & dal-o-k & `song'\\\hline
    b.& dal-o-m & dal-a-i-m & `my song(s)'\\
    c.& dal-o-d & dal-a-i-d & `your.\gsc{SG} song(s)'\\
    d.& dal-a-Ø & dal-a-i-Ø & `his/her song(s)'\\
    e.& dal-u-nk & dal-a-i-nk & `our song(s)'\\
    f.& dal-o-tok & dal-a-i-tok & `your.\gsc{PL} song(s)'\\
    g.& dal-u-k & dal-a-i-k & `their song(s)'\\
    \end{tabular}
\xe


\ex \begingl
    \gla az (\H{o}) csont -ja \textbf{-i} -Ø//
    \glb the her bone -\gsc{poss} -\gsc{PL} -\gsc{3sg}//
    \glft `her bones'//
    \endgl
\xe

Possessor agreement:
\pex 
    \a \begingl
    \gla az (én) csont -Ø -om//
    \glb the my bone -\gsc{poss} -\gsc{1sg}//
    \glft `my bone'//
    \endgl

    \a \begingl
    \gla az (én) csont -ja \textbf{-i} -m//
    \glb the her bone -\gsc{poss} -\gsc{PL} -\gsc{1sg}//
    \glft `her bones'//
    \endgl
\xe

\pex
    \a \gsc{pl} \lra{} -((j)a)i- / \trace{} \gsc{Poss}
    \a \gsc{pl} \lra{} -(V)k
    \a \gsc{Agr[1sg]} \lra{} -m
\xe



\section{*ABA (incomplete)}

    \subsection{Comparatives and superlatives}
    
    \subsection{Extending to inchoatives}
    
    \subsection{Extending to case and pronouns}
Smith et al

    \subsection{More}
McFadden, \cite{mueller20}, Andersson.


\section{Argument structure}
	\subsection{Preliminaries}
	    \subsubsection{Valency}
Verbs seems to differ in terms of the number of semantic dependents they require (or are compatible with).
\ex It rained.
\xe
\pex
	\a Chris ran.
	\a The door opened.
\xe
\pex
	\a Chris hit the ball.
	\a Chris knows the answer.
\xe
\pex
	\a John put the book on the table.
	\a \ljudge{*} John put the book.
	\a \ljudge{*} John put on the table.
	\a John gave Mary a book.
	\a \ljudge{*} John gave Mary.
	\a \ljudge{*} John gave a book.
\xe

More than 3 dependents? Transactional verbs seem to imply 3 complements semantically but don’t require that all be expressed:
\ex John bought a car (from Bill) (for \$40,000).
\xe

\textbf{Observation:} Verbs (in English) require at most one external argument and two complements.

	    \subsubsection{Unaccusativity}
Direct objects with \textbf{resultatives}:
\pex
	\a I hammered the metal flat.
	\a John stripped himself naked.
\xe

Are all intransitive verbs alike?
\pex
	\a The water froze solid.
	\a The box opened wide.
\xe
\pex
	\a John ran naked.
	\a John cried tired.
\xe

But:  John ran his shoes ragged, John cried himself tired, Babe Ruth homered his way to the hearts of America.

\pex
	\a I broke the vase.
	\a The vase broke.
\xe
\pex
	\a I ran the dishwasher.
	\a \ljudge{*} The dishwasher ran.
\xe

But: The dishwasher ran all night.

\begin{tabular}{l|l}
Unaccusatives	& Unergatives\\\hline
Underlying object	& Underlying subject\\
\cmark Resultatives	& \xmark Resultatives\\
\cmark Causative-inchoative alternations	& \xmark Causative alternation\\
\xmark \emph{Way} and reflexive constructions & \cmark \emph{Way} and reflexive constructions\\
\cmark Impersonal passives & \xmark Impersonal passives\\
\end{tabular}
\bigskip

\pex
	\a \ljudge{*} The water froze its way to Greenland.
	\a \ljudge{*} The water froze itself solid.
\xe
\pex
	\a I wrote my way out.
	\a I wrote myself to becoming Secretary of the Treasury.
\xe

\pex 
	\a \begingl
		\gla Es wurde von allen geschlafen//
		\glb it was by all slept//
		\glft `It was slept by all (here).'//
	\endgl
	\a \ljudge{*} \begingl
		\gla Es wurde von allen eingeschlafen//
		\glb it was by all fallen.asleep//
		\glft (int. `It was fallen asleep by all')//
	\endgl
\xe

\emph{habe geschlafen, bin eingeschlafen.}

	    \subsubsection{Syntactic structure and affixation}
Morphology is sensitive to these distinctions.
\pex Unaccusative vs unergative:
	\a The door re-opened.
	\a \ljudge{*} John re-laughed.
\xe

\pex Transitive vs unergative:
	\a John re-painted the wall.
	\a \ljudge{*} John re-ran.
\xe

\pex Transitive vs multiple complements:
	\a John re-painted the wall.
	\a \ljudge{*} John re-gave Bill the present.
	\a \ljudge{*} John re-put the book on the table.
\xe

    \subsection{Passivization}
    	\subsubsection{English}
Passivization changes a transitive structure into an unaccusative structure.
\pex
	\a John kissed Mary.
	\a Mary was kissed by John.
\xe

\pex
	\a John considered there to be too many	students in the class / it to be raining.
	\a 	There were considered to be too many students in the class / It was considered to be raining.
\xe

\pex
	\a John put the book on the table / gave Mary a book.
	\a The book was put on the table / Mary was given a book.
\xe

\pex
	\a John said that the book was expensive.
	\a The book was said to be expensive.
\xe

\pex
	\a John slept in the bed.
	\a The bed was slept in.
\xe

	    \subsubsection{Crosslinguistically}
\ex \begingl
	\gla Die Schlange frisst den Frosch//
	\glb the.\gsc{NOM.F} snake devours the.\gsc{ACC.M} frog//
	\glft `The snake is eating the frog'//
	\endgl
\xe

\ex \begingl
	\gla Der Frosch wird \emph{(}von der Schlange\emph{)} gefressen//
	\glb the.\gsc{NOM.M} frog Pass \phantom{(}from the.\gsc{DAT.F} snake\phantom{)} devoured//
	\glft `The frog is being eaten (by the snake)'//
	\endgl
\xe

\ex \ljudge{*} Den Frosch wird gefressen
\xe

\textbf{Five canonical properties of the passive:}
\begin{enumerate} \tightlist
	\item  The internal argument gets \emph{promoted} to subject instead of the external argument. Controls agreement.
	\item The external argument becomes an optional argument of a \emph{by}-phrase.
	\item Passive auxiliary (\emph{be}, \emph{werden}).
	\item Passive participial morphology on V (\emph{eaten}, \emph{gefressen}).
	\item Case. Accusative is \emph{absorbed} in the passive: no \gsc{ACC} assigned.
\end{enumerate}

%\begin{itemize*}
%	\item Passivization involves the \emph{suppression} of the external argument (agent, causer, experiencer).
%	\item If the structure is initially transitive, as in English, an underlying object may become the subject of the passive.
%	\item The external argument may or may not be expressible as an ``adjunct'' (in English: in a \emph{by}-phrase).
%\end{itemize*}


	    \subsubsection{Analysis}
The external argument is introduced by a separate head.
\begin{itemize} \tightlist
	\item It appears that a predicate has selectional requirements for its complement, but less so for its subject.
	\item \cite{chomsky95}: little \emph{v}.\\
		\cite{kratzer96,marantz97,pylkkanen08,layering15,kastner20ogs}: Voice.
\end{itemize}

\pex
	\a throw a baseball
	\a throw a boxing match
	\a throw a party
	\a throw a tantrum
\xe

\pex
	\a take a book from the shelf
	\a take a bus
	\a take a nap
\xe

\ex \emph{Mia left Seb}:\\
\Tree
[.TP
	\qroof{\tikz{\node (SpecTP) {\emph{Mia}};}}.DP
	[.T'
		[.T ]
		[.VoiceP
			\qroof{\tikz{\node (SpecVoice) {\emph{\sout{Mia}}};}}.DP
			[.
				[.Voice ]
				[.vP
					[.v
						[.\root{\gsc{LEAVE}} ]
						[.v ]
					]
					\qroof{\emph{Seb}}.DP
				]
			]
		]
	]
]
\begin{tikzpicture}[overlay]
	\draw[thick,->] (SpecVoice) .. controls +(south west:2) and +(south west:2) .. node[below]{\scriptsize (EPP)\phantom{aa}}(SpecTP);
\end{tikzpicture}
\xe
\bigskip

\textbf{Hypothesis:} passive morphology is the realization of a Voice head that may or may not project the external argument.
\ex \emph{Seb was left by Mia}:\\
\Tree
[.TP
	\qroof{\tikz{\node (SpecTP) {\emph{Seb}};}}.DP
	[.T'
		[.T\\\emph{was} ]
		[.VoiceP
			[.VoiceP
				[.Voice\\{[Pass]} ]
				[.vP
					[.v
						[.\root{\gsc{LEAVE}} ]
						[.v ]
					]
					\qroof{\tikz{\node (IA) {\emph{\sout{Seb}}};}}.DP
				]
			]
			[.PP
				[.P\\\emph{by} ]
				\qroof{\emph{Mia}}.DP
			]
		]	
	]
]
\begin{tikzpicture}[overlay]
	\draw[thick,->] (IA) .. controls +(south west:3) and +(south west:3) .. (SpecTP);
\end{tikzpicture}
\xe
\bigskip


\begin{mdframed}[hidealllines=true,innerleftmargin=3pt,leftmargin=0pt,innerrightmargin=10pt,rightmargin=-3pt,backgroundcolor=gray!20]
\textbf{Mystery:} Why does English (and other languages) care about the existence of a complement to the passive verb?
\pex
	\a It was danced all night long.
	\a \ljudge{*} There was danced all night long.
	\a \ljudge{*} At the party was danced by the children.
\xe
\end{mdframed}


    \subsection{Applicatives}
Applicative constructions ``add'' a complement. We've already seen some in affix order.

	    \subsubsection{Benefactives}
\pex Chichewa benefactives:
	\a \begingl
		\gla Mavuto ana- umb -a mtsuko//
		\glb Mavuto Past mold Asp waterpot//
		\glft `Mavuto molded the waterpot.'//
	\endgl
	
	\a \begingl
		\gla Mavuto ana- umb \textbf{-ir} -a \textbf{mfumu} mtsuko//
		\glb Mavuto Past mold \gsc{APPL} Asp chief waterpot//
		\glft `Mavuto molded the waterpot for the chief.'//
	\endgl
\xe
\bigskip

\pex
	\a \ljudge{*} \begingl
		\gla mkango uku- jend -er -a anjani//
		\glb lion Pres walk \gsc{APPL} Asp baboons//
		\glft (int. `The lion is walking for the baboons')//
	\endgl
	
	\a \ljudge{*} \begingl
		\gla kalulu ana- sek -er -a atsikana//
		\glb hare Past laugh \gsc{APPL} Asp girls//
		\glft (int. `The hare laughed for the girls')//
	\endgl
\xe

Draw low Appl!

In many languages, applicatives want (mono-)transitive verbs stems. In other languages this is OK.

\bigskip
Are applied arguments actual arguments or adjuncts? We'd want to see how they behave with respect to syntactic processes typical of arguments:
\begin{itemize} \tightlist
    \item Controlling agreement.
    \item Undergoing passivization - up next.
\end{itemize}

	\subsubsection{Passivized applicatives}
\pex
	\a \begingl
		\gla ndi- na- phik -ir -a atsikana nsima//
		\glb I Past cook \gsc{APPL} Asp girls cornmush//
		\glft `I cooked cornmush for the girls.'//
	\endgl

	\a \begingl
		\gla \textbf{atsikana} ana phik -ir \textbf{-idw} -a nsima//
		\glb girls Past cook \gsc{APPL} \gsc{PASS} Asp cornmush//
		\glft `The girls were cooked cornmush.'//
	\endgl

	\a \ljudge{*} \begingl
		\gla \textbf{nsima} ina phik -ir \textbf{-idw} -a atsikana//
		\glb cornmush Past cook \gsc{APPL} \gsc{PASS} Asp girls//
		\glft (Int.~`Cornmush was cooked for the girls')//
	\endgl
\xe

Predicted? Yes, if the added argument is higher than the internal argument.

Difference between languages that allow benefactives on intransitives and those that don’t?

\cite{pylkkanen08} on high and low applicatives. 


    \subsection{Causativization}
How do we express causation?

What do we think is going on in each case?
\pex
    \a Destroy the tower. (transitive/causative)
    \a Open the door. (labile/alternating)
    \a Make Sam dance. (periphrastic)
\xe

	    \subsubsection{Kinds of causatives}

Thinking in theoretical terms:
\begin{enumerate} \tightlist
	\item Attaching to a root as a verbalizer, create lexical causatives that relate to unaccusatives.\\
		\emph{open the door} %open door
	\bigskip
	
	\item Attaching to a verb, create causatives where there is a separation between a causing event and a caused event. %will see soon
	\bigskip
	
	\item Attaching to a VoiceP with an external argument.\\
		\emph{make John eat dinner} %made eat
\end{enumerate}


        \subsubsection{Morphological evidence for causativization}
Quechua (Neil Myler p.c.):
\pex
    \a \begingl
    \gla Yaku t’impu-rqa//
    \glb Water boil-PAST//
    \glft `The water boiled.'//
    \endgl
    
    \a \begingl
    \gla Wayna yaku-ta t’impu-\textbf{chi}-rqa//
    \glb boy water-ACC boil-CAUS-PAST//
    \glft `The boy boiled the water.'//
    \endgl
\xe

\ex \begingl
    \gla Marya Juan-man libru-ta rikhu-\textbf{chi}-rqa//
    \glb Mary Juan-DAT book-ACC see-CAUS-PAST//
    \glft `Mary showed John the book.'//
\endgl
\xe

What might be going on here?
\pex
    \a hang $\sim$ hung
    \a hang $\sim$ hanged
\xe
\pex
    \a rise $\sim$ rose
    \a raise $\sim$ raised
\xe

There is a generalization here. What is it? Does our current system predict it?

Overt v leads to regular inflection. Make sense if overt v intervenes between \root{\gsc{root}} and T!

        \subsubsection{Anticausativization?}
What might be going on in these Quechua examples?
\pex \a \begingl
    \gla Wayna qeru-ta p’aki-rqa//
    \glb Boy glass-ACC break-PAST//
    \glft `The boy broke the glass.'//
    \endgl
    \a \begingl
    \gla Qeru p’aki-\uline{ku}-rqa//
    \glb Glass break-REFL-PAST//
    \glft `The glass broke.'//
    \endgl
\xe
\pex \a \begingl
    \gla Juan Marya-ta phiña-\textbf{chi}-rqa//
    \glb John Mary-ACC anger-CAUS-PAST//
    \glft `John angered Mary/made Mary angry.'//
    \endgl
    \a \begingl
    \gla Marya phiña-\uline{ku}-rqa//
    \glb Mary anger-REFL-PAST//
    \glft `Mary got angry.'//
    \endgl
\xe

        \subsection{Causativizing larger events}
What's going on here? How does case get assigned?
\ex \begingl
    \gla Juan puñu-n//
    \glb Juan sleep-3SUBJ//
    \glft `Juan sleeps.'//
    \endgl
\xe
\ex \ljudge{*} \begingl
    \gla Maria Juan-ta puñu-n//
    \glb Maria Juan-ACC sleep-3SUBJ//
    \glft (int.~‘Maria sleeps Juan.’)//
    \endgl
\xe
\ex \begingl
    \gla Maria Juan-ta puñu-\textbf{chi}-n//
    \glb Maria Juan-ACC sleep-CAUS-3SUBJ//
    \glft ‘Maria makes Juan sleep.’//
    \endgl
\xe

What do we think would happen if the embedded verb were transitive?

\pex \a \begingl
    \gla Alqo wayna-ta khani-rqa//
    \glb Dog boy-ACC bite-PAST//
    \glft ‘The dog bit the boy.’//
    \endgl
    
    \a \begingl
    \gla Juan alqo-wan wayna-ta khani-\textbf{chi}-rqa.//
    \glb Juan dog-INSTR boy-ACC bite-CAUS-PAST//
    \glft ‘Juan made the dog bite the boy.’//
    \endgl
\xe

Something like `Juan made-bite the boy using the dog'.

Languages differ in how they set up such embedded causatives.


	    \subsubsection{Causativization in Hiaki \citep{harley13lingua}}
\begin{itemize} \tightlist
	\item \textbf{\emph{-tua}} takes a voiceP complement, so the external argument of the caused verb phrase is a complement of the derived causative verb.
	\item \underline{\emph{-tevo}} takes a vP complement, so the external argument of the caused verb is not expressed.
\end{itemize}

\ex \begingl
	\gla Nee usi-ta avion-ta ni'i-\textbf{tua}-ria-k//
	\glb I child-\gsc{ACC} plane-\gsc{ACC} fly-\gsc{CAUS}-\gsc{APPL}-Perf//
	\glft `I made the (model) plane fly for the child.'//
	\endgl
\xe
\includegraphics[width=0.7\textwidth]{figs/harley1.jpg}

\ex \begingl
	\gla Nee uka avion-ta ni'i-\textbf{tua}-\underline{tevo}-k//
	\glb I the plane-\gsc{acc} fly-\gsc{caus-caus.indir}-Perf//
	\glft `I had the plane flown (by somebody)'.//
	\endgl
\xe
\includegraphics[width=0.7\textwidth]{figs/harley2.jpg}	

	    \subsubsection{Passivized causatives in Hiaki}
When passive is added to \textbf{\emph{–tua}}, the external argument of the caused vP is made the subject of the passive verb, since it is an argument of the causative verb.

\ex \raisebox{-19em}{ \includegraphics[width=0.7\textwidth]{figs/harley3.jpg} }
\xe

When passive is added to \underline{\emph{-tevo}}, the object of the lower vP is made the subject of the passive, since it is the highest complement of the derived causative verb.
\ex	\raisebox{-14em}{ \includegraphics[width=0.8\textwidth]{figs/harley4.jpg} }
\xe


    \subsection{Head movement}
\textbf{Adverbs}
\ex \begingl
		\gla Jean a \fbox{souvent} embrass\'e Pierre//
		\glb Jean has often kissed Pierre//
		\glft `Jean has often kissed Pierre.'//
	\endgl
\xe

\pex
	\a \begingl
		\gla Jean \textbf{embrasse} \fbox{souvent} Pierre//
		\glb Jean kisses often Pierre//
		\glft `Jean often kisses Pierre.'//
	\endgl

	\a \ljudge{*} \begingl
		\gla Jean \fbox{souvent} \textbf{embrasse} Pierre//
		\glb Jean often kisses Pierre//
		\glft (Intended: `Jean often kisses Pierre')//
	\endgl
\xe	
\bigskip

\textbf{Negation}
\ex
	\begingl
		\gla Jean (n')a \fbox{pas} embrass\'e Pierre//
		\glb Jean \phantom{(n')}has not kissed Pierre//
		\glft `Jean has not kissed Pierre.'//
	\endgl
\xe

\pex	
	\a \begingl
		\gla Jean (n')\textbf{embrasse} \fbox{pas} Pierre//
		\glb Jean \phantom{(n')}kisses not Pierre//
		\glft `Jean does not kiss Pierre.'//
	\endgl

	\a \ljudge{*} \begingl
		\gla Jean (ne) \fbox{pas} \textbf{embrasse} Pierre//
		\glb Jean {} not kiss Pierre//
		\glft (Int.~`Jean does not kiss Pierre')//
	\endgl
\xe

Draw a tree!

Common assumptions:
\begin{itemize} \tightlist
	\item Morphologically complex words are built by head-to-head(-to-head) movement up the tree.
	\item If head-movement adjoins heads to the left of what they're adjoined to, then as you move rightwards, you go higher in scope.
	\item Some prefixes, e.g.~English \emph{re-} and \emph{un-}, would need to head-move from a low position.
\end{itemize}
Not necessarily true! But productive.


    \subsection{Cliticization}
Examples of clitics?

In some cases, morphemes seem to be only partially positioned by the syntax.
\begin{itemize} \tightlist
	\item \textbf{Clitics} are positioned with respect to phrases by the syntax.
	\item Their \emph{hosts}---the words with which they are pronounced---appear to be chosen by phonological considerations.
	\item Some clitics seem to lean on neighboring words after positioning via the syntax.
	\item Other clitics infix morphophonologically into constituents they are positioned next to.
\end{itemize}

	    \subsubsection{Straightforward clitics}
\ex Queen of England's hat\\
\Tree
[.DP
	\qroof{Queen of England}.DP
	[.D'
		[.D\\\emph{'s} ]
		[.NP\\\emph{hat} ]
	]
]
\xe

	    \subsubsection{Phonological attachment}
Where can BCS auxiliares appear?

\pex
	\a \begingl
		\gla koji \textbf{je} \v{c}ovjek koju knjigu kupio?//
		\glb which is man which book bought//
		\glft `Which man bought which book?'//
	\endgl
	
	\a \begingl
		\gla koji \v{c}ovjek \textbf{je} koju knjigu kupio?//
		\glb which man is which book bought//
		\glft `Which man bought which book?'//
	\endgl
\xe

\pex
	\a \begingl
		\gla moja mladja sestra \textbf{\'ce} do\'ci u utorak//
		\glb my younger sister will come on Tuesday//
		\glft `My younger sister will come on Tuesday.'//
	\endgl
	
	\a \begingl
		\gla moja \textbf{\'ce} mladja sestra do\'ci u utorak//
		\glb my will younger sister come on Tuesday//
		\glft `My younger sister will come on Tuesday.'//
	\endgl
\xe
After the first phonological word \textbf{or} first constituent.

\bigskip

Latin:
\pex 
	\a \begingl
		\gla {[} [ bon-i\textipa{:} puer-i\textipa{:} ] [ que [ bon-ae puell-ae ] ] ]//
		\glb good-\gsc{M.PL} boy-\gsc{M.PL} and good\gsc{F.PL} girl-\gsc{F.PL}//
	\endgl
	\a \emph{boni\textipa{:} pueri\textipa{:} bonae-que puellae}
\xe
	

	    \subsubsection{Non-active Voice}
Non-active Voice is often marked by an element external to tense and agreement.
\pex French
	\a \begingl
		\gla Jean a cassé le verre//
		\glb Jean has broken the glass//
		\glft `John has broken the glass.'//
	\endgl
	\a \begingl
		\gla Le verre \textbf{s}'est cass\'e//
		\glb the glass \gsc{SE}=has broken//
		\glft `The glass broke.'//
	\endgl
\xe

\pex Russian
	\a \begingl
		\gla ja zak\'ont\textipa{S}il klas//
		\glb I ended class//
		\glft `I ended the class.'//
	\endgl
	
	\a \begingl
		\gla klas zakont\textipa{S}il\textbf{as\super{j}}//
		\glb class end=\gsc{SJA}//
		\glft `The class was ended', `The class ended.'//
	\endgl
\xe

\ex \begingl
	\gla letom razmer belyx pjaten na Marse umen'\textipa{S}aet\textbf{sja}//
	\glb in-summer size of-white spots on Mars decrease=\gsc{SJA}//
	\glft `In the summer, the size of the white spots on Mars decreases'//
	\endgl
\xe

\pex Icelandic
	\a J\'on se\textbf{st}\\
	`John seats himself' (reflexive)
	\a J\'on gref\textbf{st}\\
	`John gets buried' (passive)
	\a Dyrnar opna\textbf{st}\\
	`The door opens' (unaccusative)
\xe

The EA is not projected.

Part of the verb and not of the next phonological word.
\pex
	\a \emph{mer-st} [m\textbf{e}r-st] `gets crushed'\\
		\emph{mer storan} [m\underline{e\textipa{:}}r] `crush big'
	\a \emph{gref-st} [kr\textbf{e}f-st] `gets buried'\\
		\emph{gref storan} [gr\underline{e\textipa{:}v}] `bury big'
\xe
\bigskip

Tree: Clitics that originate in the specifier of voice, rather than as the head of voice. They cliticize to the inflected verb.

    \subsection{For reflection}
What does the interaction of argument structure and morphology teach us about the interface between syntax and morphology?


\bibliographystyle{linquiry2}
\bibliography{lingxbib}

\end{document}
