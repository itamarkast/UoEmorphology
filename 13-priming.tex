\documentclass[a4paper,12pt]{article}

%\usepackage{times}
%\usepackage[T3,T1]{fontenc}
%\usepackage[utf8x]{inputenc}
% \usepackage[T1]{fontenc}
% \usepackage{mathptmx}  

\usepackage{fontspec} %,xunicode} % xltxtra
    \defaultfontfeatures{Ligatures=TeX}
    \setmainfont{Brill}[
    Extension=.ttf,
    UprightFont=*-Roman,
    BoldFont=*-Bold,
    ItalicFont=*-Italic,
    BoldItalicFont=*-Bold-Italic,
    Renderer=ICU]
% 	\usefonttheme{serif}
%	\setmainfont[Renderer=ICU]{Charis SIL}
% 	\setmainfont[Renderer=ICU]{Brill}

\usepackage{amsmath,amssymb} % for $\text{}$
\usepackage{url,natbib} 
\usepackage[dvipsnames]{xcolor} %,bm}
\usepackage{geometry,vmargin,setspace}
\usepackage{multirow}
\setmarginsrb{1in}{1in}{1in}{1in}{13.6pt}{0.1in}{0.1in}{0.2in}
% pdflscape} %rotating
    \setlength{\parindent}{0pt}

\usepackage{stmaryrd}
\usepackage{wasysym} %checkbox
\usepackage[normalem]{ulem} % for \sout{}
% \usepackage{mdwlist} % for \begin{itemize*} - but prefer \tightlist
\providecommand{\tightlist}{%
	\setlength{\itemsep}{0pt}\setlength{\parskip}{0pt}}

\usepackage{natbib}
	\bibpunct[:]{(}{)}{;}{a}{}{,}
	\setlength{\bibsep}{0pt plus 0.3ex}

\usepackage{longtable}
\usepackage[linkcolor=purple,citecolor=ForestGreen,colorlinks=true,urlcolor=gray,pagebackref=false]{hyperref}
\usepackage{multicol}
\usepackage{dashrule} %\hdashrule
\usepackage{array} % >{\...} in tabulars
\usepackage{arydshln}

\usepackage[framemethod=tikz,footnoteinside=false]{mdframed} 

\usepackage{expex} %Linguistics examples
\lingset{aboveexskip=0ex,belowexskip=0.5ex,aboveglftskip=-0.3em,interpartskip=0ex,labelwidth=!6pt,belowpreambleskip=0.1ex} %, *=*?}
%	\lingset{aboveexskip=0.5ex,belowexskip=0.5ex,*=??}

\usepackage[nocenter]{qtree} % trees
\usepackage{tikz}
    \tikzstyle{every picture}+=[remember picture]

\usepackage{tipa} % convenient for \textsubarch etc.

\newcommand\trace{\rule[-0.5ex]{0.5cm}{.4pt}}
\bibpunct[:]{(}{)}{;}{a}{}{,}
	\setlength{\bibsep}{0pt plus 0.3ex}

\usepackage{pifont}
\newcommand{\cmark}{\ding{51}}% 52
\newcommand{\xmark}{\ding{55}}%
 \newcommand{\hand}{\ding{43}}

\newcommand\zero{\O{}}
\newcommand\itp[1]{\textit{\textipa{#1}}}
\newcommand\gsc[1]{\textsc{\lowercase{#1}}} %for glossing in small caps - comment out to return to caps
\renewcommand\root[1]{$\sqrt{\text{#1}}$}

\newcommand\blue[1]{\textcolor{blue}{#1}}
\newcommand\red[1]{\textcolor{red}{#1}}
\newcommand\green[1]{\textcolor{ForestGreen}{#1}}
\newcommand\gray[1]{\textcolor{gray}{#1}}
\newcommand\denote[1]{$\llbracket$#1$\rrbracket$}
\newcommand\lra{$\leftrightarrow$}

\newcommand\vz{\text{Voice$_{\text{\{--D\}}}$}}
\newcommand\vd{\text{Voice$_{\text{\{+D\}}}$}}
\newcommand\pz{\text{$p_{\text{\zero}}$}}
\newcommand\va{\root{\gsc{ACTION}}}
\newcommand{\tkal}{\emph{XaYaZ}}
\newcommand{\tpie}{\emph{XiY̯eZ}}
\newcommand{\tpua}{\emph{XuY̯aZ}}
\newcommand{\thif}{\emph{heXYiZ}}
\newcommand{\thuf}{\emph{huXYaZ}}
\newcommand{\thit}{\emph{hitXaY̯eZ}}
\newcommand{\tnif}{\emph{niXYaZ}}
\newcommand\dgs[1]{\textsubarch{#1}}
\newcommand\del[1]{\sout{#1}}

\makeatletter %Allow superscript ^ and subscript _
\catcode`_=\active%
\gdef_#1{\ensuremath{{}\sb{#1}}}%
\catcode`^=\active%
\gdef^#1{\ensuremath{{}\sp{#1}}}%
\makeatother

\begin{document}
% \pagenumbering{gobble}
% \singlespacing


\hfill \emph{Morphology 2024-25, Edinburgh, Handout 13}
\bigskip


\subsection*{Step 1: Predictions}

So far, we've established that \textbf{masked priming} obtains between \textbf{morphologically} related forms:

    \begin{tabular}{lll@{ --- }l}
		\cmark & \textbf{Identity:} & church & CHURCH\\
		\cmark & \textbf{Morpho + Sem + Form:} & adapt\blue{er} & ADAPT\blue{ABLE}\\\hdashline
		\xmark & Sem + Form: & screech & SCREAM\\
		\xmark & Sem: & cello & VIOLIN\\
		\xmark & Form: & typhoid & TYPHOON\\
	\end{tabular}

\bigskip
In a follow-up study, the authors used a slightly different set of conditions. For example, they didn't include the identity condition. Here are some of the materials, divided into three conditions.
\begin{enumerate} \tightlist
    \item Figure out what each of the three conditions is supposed to be testing (give it a label).
    \item Predict whether priming obtained or not (for each of the three conditions as a whole, not for each item).
\end{enumerate}

\begin{center}
\begin{tabular}{lll}
Prime	&	Target	&		\\\hline
brothel	&	BROTH	&		\\
warfare	&	BROTH	&		\\
					
studio	&	STUD	&		\\
gently	&	STUD	&		\\
					
twinkle	&	TWIN	&		\\
cheaply	&	TWIN	&		\\
					
demonstrate	&	DEMON	&		\\
instruction	&	DEMON	&		\\
					
electron	&	ELECT	&		\\
suburban	&	ELECT	&		\\\hline
					
					
employer	&	EMPLOY	&		\\
addition	&	EMPLOY	&		\\
					
teacher	&	TEACH	&		\\
finally	&	TEACH	&		\\
					
soften	&	SOFT	&		\\
heroic	&	SOFT	&		\\
					
legendary	&	LEGEND	&		\\
anxiously	&	LEGEND	&		\\\hline

coaster	&	COAST	&		\\
muffler	&	COAST	&		\\
					
number	&	NUMB	&		\\
really	&	NUMB	&		\\
					
planet	&	PLAN	&		\\
editor	&	PLAN	&		\\
					
signet	&	SIGN	&		\\
frosty	&	SIGN	&		\\
					
rational	&	RATION	&		\\
steadily	&	RATION	&		\\			
\end{tabular}
\end{center}

% \begin{tabular}{p{4cm}c}
%     Condition & Priming (Y/N)?\\\hline
%     \\ \hline
%     \\ \hline
%     \\ \hline
% \end{tabular}

\newpage
\subsection*{Step 2: Findings}
Here is the same table from before, with the RTs (response times) for each item. Do these findings align with your predictions? Why yes/no? What do you think is going on?

\begin{center}
\begin{tabular}{lll}
Prime	&	Target	&	RT	\\\hline
brothel	&	BROTH	&	739	\\
warfare	&	BROTH	&	706	\\
					
studio	&	STUD	&	677	\\
gently	&	STUD	&	651	\\
					
twinkle	&	TWIN	&	571	\\
cheaply	&	TWIN	&	577	\\
					
demonstrate	&	DEMON	&	585	\\
instruction	&	DEMON	&	555	\\
					
electron	&	ELECT	&	623	\\
suburban	&	ELECT	&	620	\\\hline
					
					
employer	&	EMPLOY	&	564	\\
addition	&	EMPLOY	&	606	\\
					
teacher	&	TEACH	&	538	\\
finally	&	TEACH	&	559	\\
					
soften	&	SOFT	&	538	\\
heroic	&	SOFT	&	569	\\
					
legendary	&	LEGEND	&	523	\\
anxiously	&	LEGEND	&	589	\\\hline

coaster	&	COAST	&	555	\\
muffler	&	COAST	&	614	\\
					
number	&	NUMB	&	610	\\
really	&	NUMB	&	659	\\
					
planet	&	PLAN	&	527	\\
editor	&	PLAN	&	603	\\
					
signet	&	SIGN	&	535	\\
frosty	&	SIGN	&	549	\\
					
rational	&	RATION	&	616	\\
steadily	&	RATION	&	640	\\
\end{tabular}
\end{center}



% \bibliographystyle{linquiry2}
% \bibliography{lingxbib}

\end{document}