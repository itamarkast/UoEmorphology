\documentclass[a4paper,12pt]{article}

%\usepackage{times}
%\usepackage[T3,T1]{fontenc}
%\usepackage[utf8x]{inputenc}
% \usepackage[T1]{fontenc}
% \usepackage{mathptmx}  

%\usepackage{fontspec} %,xunicode} % xltxtra
    %\defaultfontfeatures{Ligatures=TeX}
   % \setmainfont{Brill}[
  %  Extension=.ttf,
  %  UprightFont=*-Roman,
  %  BoldFont=*-Bold,
  %  ItalicFont=*-Italic,
  %  BoldItalicFont=*-Bold-Italic,
 %   Renderer=ICU]
% 	\usefonttheme{serif}
%	\setmainfont[Renderer=ICU]{Charis SIL}
% 	\setmainfont[Renderer=ICU]{Brill}

\usepackage{amsmath,amssymb} % for $\text{}$
\usepackage{url,natbib} 
\usepackage[dvipsnames]{xcolor} %,bm}
\usepackage{geometry,vmargin,setspace}
\usepackage{multirow}
\setmarginsrb{1in}{1in}{1in}{1in}{13.6pt}{0.1in}{0.1in}{0.2in}
% pdflscape} %rotating
    \setlength{\parindent}{0pt}

\usepackage{stmaryrd}
\usepackage{wasysym} %checkbox
\usepackage[normalem]{ulem} % for \sout{}
% \usepackage{mdwlist} % for \begin{itemize*} - but prefer \tightlist
\providecommand{\tightlist}{%
	\setlength{\itemsep}{0pt}\setlength{\parskip}{0pt}}

\usepackage{natbib}
	\bibpunct[:]{(}{)}{;}{a}{}{,}
	\setlength{\bibsep}{0pt plus 0.3ex}

\usepackage{longtable}
\usepackage[linkcolor=purple,citecolor=ForestGreen,colorlinks=true,urlcolor=gray,pagebackref=false]{hyperref}
\usepackage{multicol}
\usepackage{dashrule} %\hdashrule
\usepackage{array} % >{\...} in tabulars
\usepackage{arydshln}

\usepackage[framemethod=tikz,footnoteinside=false]{mdframed} 

\usepackage{expex} %Linguistics examples
\lingset{aboveexskip=0ex,belowexskip=0.5ex,aboveglftskip=-0.3em,interpartskip=0ex,labelwidth=!6pt,belowpreambleskip=0.1ex} %, *=*?}
%	\lingset{aboveexskip=0.5ex,belowexskip=0.5ex,*=??}

\usepackage[nocenter]{qtree} % trees
\usepackage{tikz}
    \tikzstyle{every picture}+=[remember picture]

\usepackage{tipa} % convenient for \textsubarch etc.

\newcommand\trace{\rule[-0.5ex]{0.5cm}{.4pt}}
\bibpunct[:]{(}{)}{;}{a}{}{,}
	\setlength{\bibsep}{0pt plus 0.3ex}

\usepackage{pifont}
\newcommand{\cmark}{\ding{51}}% 52
\newcommand{\xmark}{\ding{55}}%
 \newcommand{\hand}{\ding{43}}

\newcommand\zero{\O{}}
\newcommand\itp[1]{\textit{\textipa{#1}}}
\newcommand\gsc[1]{\textsc{\lowercase{#1}}} %for glossing in small caps - comment out to return to caps
\renewcommand\root[1]{$\sqrt{\text{#1}}$}

\newcommand\blue[1]{\textcolor{blue}{#1}}
\newcommand\red[1]{\textcolor{red}{#1}}
\newcommand\green[1]{\textcolor{ForestGreen}{#1}}
\newcommand\gray[1]{\textcolor{gray}{#1}}
\newcommand\denote[1]{$\llbracket$#1$\rrbracket$}
\newcommand\lra{$\leftrightarrow$}

\newcommand\vz{\text{Voice$_{\text{\{--D\}}}$}}
\newcommand\vd{\text{Voice$_{\text{\{+D\}}}$}}
\newcommand\pz{\text{$p_{\text{\zero}}$}}
\newcommand\va{\root{\gsc{ACTION}}}
\newcommand{\tkal}{\emph{XaYaZ}}
\newcommand{\tpie}{\emph{XiY̯eZ}}
\newcommand{\tpua}{\emph{XuY̯aZ}}
\newcommand{\thif}{\emph{heXYiZ}}
\newcommand{\thuf}{\emph{huXYaZ}}
\newcommand{\thit}{\emph{hitXaY̯eZ}}
\newcommand{\tnif}{\emph{niXYaZ}}
\newcommand\dgs[1]{\textsubarch{#1}}
\newcommand\del[1]{\sout{#1}}

\makeatletter %Allow superscript ^ and subscript _
\catcode`_=\active%
\gdef_#1{\ensuremath{{}\sb{#1}}}%
\catcode`^=\active%
\gdef^#1{\ensuremath{{}\sp{#1}}}%
\makeatother

\begin{document}
% \pagenumbering{gobble}
% \singlespacing


\hfill \emph{Théories linguistiques 2024, Utrecht}
\bigskip

%La fois passée, on a examiné l'ordre des affixes dans les verbes. Faisons la même chose pour les noms maintenant.  


Une représentation possible du genre en français:
\pex
    \a \begingl
        \gla l-e chat-ø//
        \glb \gsc{déf}.\gsc{M} chat.\gsc{M}//
        \endgl
    \a \begingl
        \gla l-es chat-ø-s//
        \glb \gsc{déf}.\gsc{PL} chat-\gsc{M}-\gsc{PL}//
        \endgl
\xe
\pex
    \a \begingl
        \gla l-a chatt-e//
        \glb \gsc{déf}.F chat-F//
        \endgl
    \a \begingl
        \gla l-es chatt-e-s//
        \glb \gsc{déf}.\gsc{PL} chat-\gsc{F-PL}//
        \endgl
\xe

Dessinez un arbre (schématique) pour le nom féminin pluriel  \emph{chattes} en~(\lastx b).

%\ex \label{tree:pl-catalan}
% \Tree
% [.
    % [.
         %[.\emph{chat} ]
       %  [.\gsc{[F]} \emph{e} ]
    % ]
     %[.\gsc{[PL]} \emph{s} ]
% ]
%\xe

\vspace{3cm}
% Yupik \citep[43]{mithun99}
% \pex
%     \a yug-\textbf{pag}-\uline{cuar}\\
%     person-big-little\\
%     `little giant'
%     \a yug-\uline{cuar}-\textbf{pag}\\
%     person-little-big\\
%     `big midget'
% \xe

%Français:
\ex soci-al-is-ation \label{ex:globalization}
\xe
\ex capital-is-ation-s
\xe

Dessinez une structure pour ~(\ref{ex:globalization}):
%\ex
% \Tree
% [.[N]
    % [.[V]
        % [.[A]
            % [.[N] \emph{soci} ]
            % [.[A] \emph{al} ]
       %  ]
       %  [.[V] \emph{is} ]
    % ]
    % [.[N] \emph{ation} ]
% ]
%\xe

\vspace{5cm}

Est-ce que c'est de la morphologie ou de la syntaxe? 



\end{document}